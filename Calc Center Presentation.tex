\documentclass[unicode]{beamer}
\usepackage[russian]{babel}
\usepackage[russian]{babel}
\usepackage{amsmath,mathrsfs,amsfonts,amssymb}
\usepackage{graphicx, epsfig}
\usepackage{subfig}
\usepackage{floatflt}
\usepackage{epic,ecltree}
\usepackage{mathtext}
\usepackage{fancybox}
\usepackage{fancyhdr}
\usepackage{enumerate}
\usepackage{wrapfig}
\usepackage{array}
\usepackage{makecell}
\usepackage{colortbl}
\usepackage{caption}
\usepackage{graphicx}
\usepackage{float}
\usepackage{multicol}
\usepackage{wrapfig}
\usepackage{hhline}

\newcommand{\T}{^{\text{\tiny\sffamily\upshape\mdseries T}}}
\newcommand\argmin{\mathop{\arg\min}}
\newcommand\argmax{\mathop{\arg\max}}

\usetheme{Warsaw}
\usecolortheme{sidebartab}
\setbeamertemplate{footline}[author]
\expandafter\def\expandafter\insertshorttitle\expandafter{%
  \insertshorttitle\hfill
  \insertframenumber\,/\,\inserttotalframenumber}

\title[Медицинские распознавания]{Задача распознавания типа раковой опухоли на основе метода логистической регрессии. Методы анализа движений человека на
основе показаний сигналов и сенсоров}
\author{Никитин Артем}
\institute[МФТИ]{Московский физико-технический институт \\
    Факультет прикладной математики и информатики\\
    Кафедра интеллектуальных систем
}
\date{Москва, 2024}


\begin{document}


\begin{frame}
    \vspace{-1pt}\titlepage 
\end{frame}

\begin{frame}{План}
    \begin{multicols}{2}
        \tableofcontents
    \end{multicols}
\end{frame}

\section{Постановка задачи опухолей}

\subsection{Цель работы}
\begin{frame}{Цель работы}
    \vspace{-2pt}\begin{block}{Задача}
    \footnotesize
        Требуется разработать для медицинского учреждения, занимающегося исследованиями и леченнием раковых опухолей, алгоритм, предсказывающий тип опухоли на основе ее визуальных параметров. 
    \end{block}

    \vspace{-2pt}\begin{block}{Актуальность вопроса}
    \footnotesize
        \begin{enumerate}
            \item Задача актуальна для многих клиник как для автоматизации, так и для более точного детектирования злокачественных опухолей.
            \item Задача применима в учебных целях для квалификации молодых специалистов. Верное предсказание повышает качество лечения.
        \end{enumerate}
    \end{block}

    \vspace{-2pt}\begin{block}{Требования к модели}
    \footnotesize
        \begin{itemize}
            \item Оптимально подобранные параметры модели на основе уже собранных данных
            \item Достаточно хорошее определение типа опухоли 
        \end{itemize}
    \end{block}
\end{frame}

%\begin{frame}
%    \frametitle{Собранные данные}
%
%    Для каждого признака посчитаны: среднее значение (пример), стандартное отклонение и наихудшее (максимальное).
%    {\centering
%        {\small
%            \begin{tabular}{|p{55pt}|p{23pt}|p{141pt}|p{38pt}|}
%            \hline
%            \textbf{Признак} & \textbf{Тип} & \textbf{Описание} & \textbf{Пример}  \\
%            \hline
%            Радиус & $\mathbb{R}$ & Расстояние от центра до края & 17.99 \\
%            \hline
%            Текстура & $\mathbb{R}$ & Значения серого цвета & 10.38 \\
%            \hline
%            Периметр & $\mathbb{R}$ & Значение периметра & 122.80 \\
%            \hline
%            Площадь & $\mathbb{R}$ & Значение площади & 1001.0 \\
%            \hline
%            Гладкость & $\mathbb{R}$ & Локальные изменения радиуса & 0.11840 \\
%            \hline
%            Плотность & $\mathbb{R}$ & $\frac{Perimeter^2}{Area} - 1 $ & 0.27760 \\
%            \hline
%            Вогнутость & $\mathbb{R}$ & Коэффициент вогнутости & 0.30010 \\
%            \hline
%            Доля вогн. & $\mathbb{R}$ & Доля вогнутых участков & 0.14710 \\
%            \hline
%            Симметрич. & $\mathbb{R}$ & Коэффициент симметричности & 0.2419 \\
%            \hline
%            Детальность & $\mathbb{R}$ & Фрактальная размерность & 0.07871 \\
%            \hline
%            \textbf{Опухоль} & \textbf{B/M} & \textbf{Тип раковой опухоли} & \textbf{B}  \\
%            \hline
%            \end{tabular}
%        }
%    }
%\end{frame}

\subsection{Формализация задачи}
\begin{frame}
    \frametitle{Формализация задачи}

    \begin{block}{Дано}
        Множество признаков $X$, в нашем случае числовых непрерывных из $\mathbb{R}$, и множество меток классов Y.
    \end{block}

    \begin{block}{Требуется}
        Создать и обучить алгоритм логистической регрессии на основе имеющихся данных, подобрать оптимальные параметры, оценить точность его работы.
        
        Оценить ковариацию признаков при помощи случайных подвыборок.
    \end{block}
\end{frame}

\subsection{Описание алгоритма}
\begin{frame}{Логистическая регрессия}
    \begin{block}{Линейный классификатор $a: X \mapsto Y$}
    $$ a(x, w) = \text{sign}\left( \sum\limits_{i=1}^{n} \omega_i f_i - \omega_0 \right) = \text{sign}\langle x, \omega \rangle  $$
    \end{block}

    \begin{block}{Минимизация эмпирического риска}
        $$ Q(\omega) = \sum\limits_{i=1}^{k} \ln{1 + \exp{(-y_i \langle x_i, \omega \rangle)}} \rightarrow  \min\limits_{\omega}$$
    \end{block}

    \begin{block}{Оценка вероятности принадлежности}
        $$\mathbb{P}(y | x) = \sigma(y \langle x_i, \omega \rangle), \, \, \, \, \sigma(t) = \frac{1}{1 + e^{-t}}$$
    \end{block}
\end{frame}

\section{Эксперимент}
\subsection{Предварительная обработка}
\begin{frame}{Препроцессинг}
    \begin{itemize}
    \item Данные в выборке есть все. Размер выборки небольшой, это может стать проблемой.
    \item Признаки зависимы относительно друг друга, уберем лишнее.
    \item Поиск выбросов - нет явно выраженных далеких точек
    \item Перевод категориальных в вещественные не требуется. Целевой в $\{-1, 1\}$
    \item Стандартизация - у всех близко к нормальному распределению $\Rightarrow StandartScaler$
    \item Выбор нормы: $F_1$ норма в силу бинарной классификации и небольшого дизбаланса данных
    \end{itemize}
\end{frame}

\begin{frame}{Препроцессинг}
    \begin{figure}[!htbp]
        \begin{centering}
            \subfloat[Распределение радиуса. Красный - злокачественная]{
                \captionsetup{justification=centering}
                \includegraphics[width=0.25\textwidth]{20241102_081013.jpg}
            }
            \subfloat[Корреляционная матрица]{
                \captionsetup{justification=centering}
                \includegraphics[width=0.7\textwidth]{20241102_074615.jpg}
            }
        \end{centering}
    \end{figure}
\end{frame}

\subsection{Результаты эксперимента}
\begin{frame}{Результат работы}
    \begin{itemize}
        \item Подбираем оптимальный параметр регуляризации методом кросс-валидации. Итог: $C =  0.65$
        %\item CV: $F_1 = 0.976, Accuracy = 0.963$
        %\item Test: $F_1 = 0.969, Accuracy = 0.951$
        \item Значения весов. Свободный коэффициент порядка 1
    \end{itemize}

%    {\centering
%    {\small
%        \begin{tabular}{|p{51pt}|p{24pt}||p{51pt}|p{24pt}||p{51pt}|p{24pt}|}
%        \hline
%        \multicolumn{2}{|c||}{\textbf{Среднее}} & \multicolumn{2}{|c||}{\textbf{Отклонение}} & \multicolumn{2}{|c|}{\textbf{Худшее}} \\
%        \hline
%        \textbf{Признак} & \textbf{Вес} & \textbf{Признак} & \textbf{Вес} & \textbf{Признак} & \textbf{Вес}  \\
%        \hline
%        Радиус & -1.80 & Радиус & -1.72 &&\\
%        \hline
%        Текстура & -1.09 & Текстура & -0.17 && \\
%        \hline
%        Гладкость & 0.32 & Гладкость & 0.44 & Гладкость & -1.26 \\
%        \hline
%        Доля вогн. & -1.79 & Доля вогн. & -0.4 && \\
%        \hline
%        Симметр. & 0.09 & Симметр. & 0.35 & Симметр. & -0.99 \\
%        \hline
%        Детальн. & 0.83 & Детальн. & 0.76 & Детальн. & -0.89 \\
%        \hline
%        \end{tabular}
%    }
%}
\vspace{3pt}\hspace{13pt}
\begin{centering}
    \begin{small}
        \begin{tabular}{|p{50pt}||p{35pt}|p{35pt}||p{35pt}|p{35pt}|}
            \hline
            \textbf{Модель} & \multicolumn{2}{|c||}{\textbf{Validation}} & \multicolumn{2}{|c|}{\textbf{Test}} \\
            \hline
            & \textbf{F1} & \textbf{Acc} & \textbf{F1} & \textbf{Acc}  \\
            \hline
            kNN & 94.9 & 92.6 & 95.6 & 95.5 \\
            \hline
            SVM & 97.8 & 96.2 & 95.6 & 95.0 \\
            \hline
            LogRegress & 97.6 & 96.3 & 96.9 & 95.1 \\
            \hline
            PotenFunc & 98.0 & 96.4 & 94.1 & 94.7 \\
            \hline
        \end{tabular}
    \end{small}
\end{centering}
\vspace{5pt}

\begin{itemize}
    \item Признаки с максимальными весами $(1.7)$: средний радиус, средняя текстура, доля вогнутых, отклонение по радиусу
    \item Признаки с минимальными весами $0.09$: средняя симметрия, отклонение по текстуре
\end{itemize}
\end{frame}

\begin{frame}{Ковариация параметров}
    \begin{figure}[!htbp]
        \begin{centering}
            \subfloat[Ковариационная матрица параметров, найденная при помощи метода бутстрепа]{
                \captionsetup{justification=centering}
                \includegraphics[width=0.775\textwidth]{20241102_112902.jpg}
            }
        \end{centering}
    \end{figure}
\end{frame}

\section{Постановка задачи трекинга}
\subsection{Цель работы}
\begin{frame}{Цель работы}

    \begin{block}{Мотивация}
    \footnotesize
        Распознавание и трекинг физической активности человека. Применяется в умных часах, фитнес-браслетах, кардиодатчиках.
    \end{block}

    \begin{block}{Вопрос}
        \footnotesize
            Почему может быть не достаточно классификации или кластеризации с использованием стандартных методов машинного обучения?
    \end{block}

    \begin{block}{Ответ}
        \footnotesize
            При такой постановке задачи невозможно осуществлять учет активностей, которые не присутствовали при обучении модели. 
            Предложение: найти способ построения признакового описания временного ряда с использованием его структуры сигнала, для последующего решения задачи кластеризации на не представленных ранее активностях.
        \end{block}
\end{frame}

\subsection{Формализация задачи}
\begin{frame}{Формализация задачи}
    \begin{block}{Задача}
    \footnotesize
        Требуется: построить модель, которая задавала бы отображнеие из пространства временного ряда в пространство меньшей размерности. Требования:
        \begin{itemize}
            \item Метрика между полученными векторами представления сигналов датчика, соответствующими одинаковым видам активности, меньше, чем между соответствующими разным
            \item Оценка результата: по активностям, которых не было при обучении
            \item Рассматриваем трекинг с акселерометра и гироскопа
        \end{itemize}
    \end{block}

    \begin{block}{Предлагаемое решение}
    \footnotesize
        Нейросетевая модель для получения репрезентативного признакового описания, являющаяся необходимым отображением
    \end{block}
\end{frame}


\section{Эксперимент трекинга}
\subsection{Предварительная обработка}
\begin{frame}{Выборка}
    \begin{block}{Процесс формирования обучающей выборки}
        \begin{itemize}
            \item Данные по трем координатам с акселерометра и гироскопа объединяем в один вектор
            \item Данные рабиваем на непересекающиеся непрерывные сегменты фиксированной длины
            \item Каждому сегменту $X_i$ ставим в соответствие метку $y_i$ вида активности - получили \textbf{обучающую выборку}
        \end{itemize}
    \end{block}
        \begin{block}{}
            Тройка $-$ три сегмента временного ряда $(X_i, X_j. X_k)$, где $X_j$ не совпадает с $X_i$, но $y_j = y_i$, и при этом $y_k \neq y_i$
            $$\mathbb{L} = \sum\limits_{i, j, k}max(\|f(X_i) - f(X_j)\|^2 - \|f(X_i) - f(x_k)\|^2 + margin, 0)$$
    \end{block}
\end{frame}

\subsection{Обучение}
\begin{frame}{Модель}
    Данных много. Надо выделить существенные
    \begin{itemize}
        \item $\argmax_{X_j} \| f(x_i) - f(x_j)\|$
        \item $\argmin_{X_k} \| f(x_i) - f(x_k)\|$
    \end{itemize}
    В идеале $\| f(x_i) - f(x_j)\|^2 < \| f(x_i) - f(x_k)\|^2$
        \begin{enumerate}
        \item Выборка без части активностей $\Rightarrow$ Модель ResNetLSTM
        \item Выборка + валидация $\Rightarrow$ Модель $\Rightarrow$ Классификатор
        \item Тестовая $\Rightarrow$ Классификатор
    \end{enumerate}
    \vspace*{-15pt}\begin{figure}
         \begin{centering}
            \subfloat{
                \includegraphics[width=0.75\textwidth]{20241102_190618.jpg}
            }
        \end{centering}
    \end{figure}
\end{frame}

\subsection{Результаты эксперимента}
\begin{frame}{Результаты работы}
    \vspace{3pt}\begin{centering}
        \begin{small}
            \begin{tabular}{|p{50pt}||p{40pt}|p{40pt}||p{40pt}|p{40pt}|}
                \hline
                \textbf{Модель} & \multicolumn{2}{|c||}{\textbf{Crossfit activities}} & \multicolumn{2}{|c|}{\textbf{Daily sports activities}} \\
                \hline
                & \textbf{F1} & \textbf{Ac} & \textbf{F1}& \textbf{Ac} \\
                \hline
                kNN & 89.9 & 87.6 & 92.4 & 93.6\\
                \hline
                SVM & 88.4 & 87.2 & 90.3 & 91.2 \\
                \hline
                RandForest & 88.7 & 87,2 & 89.8 & 89.5 \\
                \hline
            \end{tabular}
        \end{small}
    \end{centering}
    \vspace{-12pt}\begin{figure}
         \begin{centering}
            \subfloat{
                \includegraphics[width=0.48\textwidth]{20241102_193450.jpg}
            }
        \end{centering}
    \end{figure}
\end{frame}

\section{Выводы}
\begin{frame}{Выводы}
    \begin{enumerate}
        \item Сформулирована задача предсказания типа раковой опухоли по ее физическим парметрам, предложен алгоритм обучения и подбора гиперпараметров, дающий оптимальные результаты.
        Алгоритм сравнен с несколькими другими. Был сделан вывод о его оптимальности на датасете Breast Cancer.
        \item Сформулирована задача постоения признакового описания данных с акселерометра и гироскопа. Предложен оптимальный алгоритм обучения и тестирования модели.
        Проведен анализ на датасетах  Daily Sports activities и Crossfit Activities Dataset.
    \end{enumerate}

    \begin{itemize}
        \item
            \textbf{Публикация по теме:} Филиппова А.\,В. Методы анализа движений человека на
            основе показаний сигналов с сенсоров. 2021.
    \end{itemize}
\end{frame}

\end{document}
