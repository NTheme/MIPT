
\begin{abstract}
    Во многих задачах графы имеют веса на ребрах, вследствие чего представляют интерес задачи о поиске кратчайших путей, минимальных циклов.
    В данной работе рассматриваются приближения решений задачи о коммивояжере, которые дают за полиномиальное время результат не хуже фиксированного ограничения сверху.
\end{abstract}
\begin{center}
    \textbf{Ключевые слова:} коммивояжер, графы, приближения, метрика, оптимизация
\end{center}

\section{Введение}

Задача коммивояжёра (TSP, Traveling Salesman Problem) является одной из ключевых задач комбинаторной оптимизации.
Её решение находит применение в логистике, проектировании систем, маршрутизации, оптимизации, 
планировании и многих других областях. 

Сложность задачи заключается в её принадлежности к классу NP-трудных задач, 
что делает поиск точного решения крайне сложным при увеличении числа вершин.

Метрическая задача коммивояжёра представляет собой особый случай общей задачи, в котором граф является полным, 
а функция весов удовлетворяет неравенству треугольника. 
Этот вариант задачи обладает рядом свойств, которые позволяют разрабатывать алгоритмы с фиксированным качеством приближения,
пусть и не самым оптимальным.

Одним из значимых достижений в этой области является разработка приближённого алгоритма Кристофидеса \cite{christofides1976},
обеспечивающий коэффициент приближения строго меньше 1.5, и различные алгоритмы на основе стандартных способов оптимизации.

Исторически проблема коммивояжёра возникла как математическая модель для оптимизации маршрутов путешественников. 
Первоначально она была предложена в XVIII веке, 
однако её систематическое исследование началось только в середине XX века. 
С тех пор было разработано множество алгоритмов и подходов, включая как точные методы, так и эвристические алгоритмы.

Данная работа посвящена анализу и реализации приближённых алгоритмов для метрической задачи коммивояжера.
В рамках исследования:
\begin{enumerate}
    \item Доказывается ограничение стандартной задачи коммивояжера.
    \item Разрабатывается алгоритм 2-приближения на основе остовного дерева.
    \item Реализуется алгоритм 1.5-приближения с использованием остовного дерева и паросочетания.
    \item Результаты опираются на классические подходы, расширяя их применение и демонстрируя эффективность в решении сложных задач оптимизации.
\end{enumerate}

\begin{definition}
    Функция $w: \, V^2 \rightarrow \mathbb{R_+}$ называется метрической, если $\forall x, y, z \in V: \, w((x, z)) \leq w((x, y)) + w((y, z))$,
    где $(u, v)$ - ребро между вершинами $u$ и $v$.
\end{definition}

\begin{definition}
    \textbf{Метрический граф} - такой полный граф $G = (V, E, w)$, у которого $w$ - метрическая функция.
\end{definition}

\begin{definition}
    Назовем \textnormal{\textbf{весом цикла}} $U = \{u_i\}_{i=1}^{n}: \, \forall i \, (u_i \in V) \wedge u_1 = u_{n+1}$ в графе $G = (V, E, w)$ число
    $\mathbf{W(U)} = \sum\limits^{n}_{i=1}w((u_i, u_{i+1}))$.
\end{definition}

\begin{definition}
    Обозначим за $\mathbf{H^*(G)}$ гамильтонов цикл минимального веса в графе $G$.
    Его вес, соответственно, равен $\mathbf{W(H^*(G))}$.
\end{definition}

\begin{definition}
    Обозначим за $\mathbf{O^*(G)}$ остовное дерево минимального веса в графе $G$.
    Его вес, соответственно, равен $\mathbf{W(O^*(G))}$.
\end{definition}

\begin{definition}
    Обозначим за $\mathbf{M^*(G)}$ полное паросочетание минимального веса в графе $G$.
    Его вес, соответственно, равен $\mathbf{W(M^*(G))}$.
\end{definition}