\section{Формулировка задачи}
\begin{definition}
    Пусть $G = (V, E)$ --- граф, где $V$ --- вершины, $E$ --- ребра.
    \textnormal{\textbf{Гамильтонов цикл}} - это замкнутый путь,
    который проходит через каждую вершину данного графа ровно по одному разу.
\end{definition}

\begin{problem}
\textnormal{\textbf{О коммивояжере.}} Дан граф $G = (V, E, w)$, где $w: E \rightarrow \mathbb{R}_{+}$ --- функция весов на ребрах.
Требуется найти гамильтонов цикл минимального веса.
\end{problem}

\begin{theorem}
    \label{theorem:1-1}
    Если существует полиномиальный алгоритм $\alpha$-приближения в задаче коммивояжера в произвольном графе,
    то \textnormal{\textbf{P}} $=$ \textnormal{\textbf{NP}}.
\end{theorem}
\begin{proof}
    Для доказательства представим алгоритм решения задачи \textbf{HAMCYCLE}, работающий за полиномиальное время,
    используя полиномиальное $\alpha$-приближение задачи о коммивояжере.
    Пусть дан граф $G = (V, E)$, в котором мы должны проверить наличие гамильтонова цикла.
    Достроим $G$ до графа $G' = (V, E', w)$ - полного взвешенного следующим образом:

    $$w(u, v) =
        \begin{cases}
            1,              & \text{если } (u, v) \in E \\
            \alpha |V| + 1, & \text{иначе}
        \end{cases}$$

    Тогда вес минимального гамильтоного цикла в $G'$ будет равен $|V|$, если в $G$ существовал гамильтонов цикл и хотя бы $\alpha |V| + 1$, иначе.
    Найдем $\alpha$-приближение задачи о коммивояжере на графе $G'$.
    Тогда если в $G$ существовал гамильтонов цикл, мы получим цикл веса не более $\alpha |V|$, а значит в нем не может оказаться ребер, которые мы добавили.
    Получается, мы тем самым нашли гамильтонов цикл в графе $G$ за полиномиальное время,
    доказав, что \textbf{P} $=$ \textbf{NP}.
\end{proof}
\begin{corollary}
    В силу теоремы~\ref{theorem:1-1} для стандартной задачи коммивояжёра не существует константных алгоритмов приближения, если \textnormal{\textbf{P}} $\neq$ \textnormal{\textbf{NP}}.
\end{corollary}