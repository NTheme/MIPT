\section{Метрические графы}

\subsection{Введение}

\begin{agreement}
    Будем рассматривать \textit{метрические графы} и, соответственно \textit{метрическую задачу коммивояжера}.
\end{agreement}

\begin{lemma}
    \label{lemma:2-1}
    Пусть $G = (V, E, w)$ --- метрический граф,
    $U = \{u_i\}_{i=1}^{n+1}: \, \forall i \, (u_i \in V) \wedge u_1 = u_{n+1}$ - цикл в $G$, не обязательно являющийся простым,
    $U' = \{u_{i_j}\}_{j=1}^{k+1}: \, \forall j \, (i_j < i_{j+1}) \wedge u_{i_1} = u_{i_{k+1}} = u_1$ --- подпоследовательность вершин $U$, также задающая цикл.
    Тогда $W(U') \leq W(U)$.
\end{lemma}
\begin{proof}
    Так как граф метрический, то:
    $$\forall j \, w((u_{i_j}, u_{i_{j+1}})) \leq \sum\limits_{k = i_j}^{i_{j+1} - 1} w((u_k, u_{k+1})).$$
    Следовательно:
    $$W(U') = \sum\limits_{j=1}^{k} w((u_{i_j}, u_{i_{j+1}})) \leq \sum\limits_{j=1}^{k} \sum\limits_{l = i_j}^{i_{j+1} - 1} w((u_l, u_{l+1})) =
        \sum\limits_{i = 1}^{n} w((u_i, u_{i+1})) = W(U)$$
\end{proof}

\begin{lemma}
    \label{lemma:2-2}
    Пусть $G = (V, E, w)$ --- метрический граф.
    Тогда $W(O^*(G)) \leq W(H^*(G))$.
\end{lemma}
\begin{proof}
    Получим гамильтонов путь $H'(G)$, удалив произвольное ребро $(u, v)$ из $H^*(G)$.
    Тогда $W(O^*(G)) \leq W(H'(G)) \leq W(H^*(G))$ в силу определения функции $W$ и так как гамильтонов путь проходит по всем вершинам.
\end{proof}

\subsection{2-приближение}
Воспользуемся алгоритмом поиска минимального остовного дерева для построения гамильтонова цикла.

Построим 2-приближение метрической задачи коммивояжера для графа $G = (V, E, w)$, воспользовавшись следующим алгоритмом:
\begin{enumerate}[label=\arabic*.]
    \item Построим $O^*(G)$ за полиномиальное время при помощи aлгоритма \textit{Прима}.
    \item Обойдем $O^*(G)$ при помощи алгоритма \textit{DFS}, выписывая вершины в список $U$ порядке посещения, за полиномиальное время.
    \item Построим гамильтонов цикл $H'$, проходящий по вершинам в порядке их первого появления в $U$, так же за полиномиальное время,
          не забыв в конец добавить первую вершину для получения цикла.
\end{enumerate}

\begin{theorem}
    \label{theorem:2-1}
    Вышеприведенный алгоритм построения 2-приближения метрической задачи коммивояжера корректен.
\end{theorem}
\begin{proof}
    Достаточно показать, что $W(H') \leq 2 W(H^*(G))$. Рассмотрим полученный список вершин $U = (u_1, \dots, u_{2|V|-1})$:
    $$\sum\limits_{i=1}^{2|V|-2} w((u_i, u_{i+1})) = 2 W(O^*(G)),$$
    так как каждое ребро будет рассмотрено дважны (при спуске и подъеме).
    Теперь расмотрим $H'(G) = (v_{i_1}, \dots, v_{i_{n+1}}): \, v_{i_1} = v_{i_{n+1}} = v_1 = v_{2|V|-1}$.

    В силу леммы~\ref{lemma:2-1}: $W(H') \leq 2 W(O^*(G))$.

    В силу леммы~\ref{lemma:2-2}: $W(O^*(G)) \leq W(H^*(G)) \Rightarrow W(H') \leq 2 W(H^*(G))$, что и требовалось.
\end{proof}

\subsection*{Реализация и время работы}

Рассмотрим поближе детали реализации.

\begin{algorithm}[H]
    \caption{Алгоритм Прима}
    \begin{algorithmic}[1]
        \State $T \gets \emptyset$
        \For{каждая вершина $i \in V$}
        \State $d[i] \gets \infty$
        \State $p[i] \gets -1$
        \EndFor
        \State $d[1] \gets 0$
        \State $Q \gets V$
        \State $v \gets \text{Extract.Min}(Q)$
        \While{$Q \neq \emptyset$}
        \For{каждая вершина $u$, смежная с $v$}
        \If{$u \in Q$ и $w(v, u) < d[u]$}
        \State $d[u] \gets w(v, u)$
        \State $p[u] \gets v$
        \EndIf
        \EndFor
        \State $v \gets \text{Extract.Min}(Q)$
        \State $T \gets T + (p[v], v)$
        \EndWhile
    \end{algorithmic}
\end{algorithm}

\begin{algorithm}[H]
    \caption{Алгоритм поиска в глубину (DFS)}
    \begin{algorithmic}[1]
        \State \text{visited} $\gets$ массив длины $n$, заполненный значениями \texttt{false}
        \State \texttt{visited}[u] $\gets$ \texttt{true}
        \For{каждое ребро $(u, v) \in G$}
        \If{not \texttt{visited}[v]}
        \State \texttt{DFS}(v)
        \EndIf
        \EndFor
    \end{algorithmic}
\end{algorithm}

Алгоритм Прима работает за $\mathcal{O}(|E| + |V| \log(|V|))$, что нетрудно увидеть из вложенности циклов.

Алгоритм $DFS$ работает за $\mathcal{O}(|E| + |V|)$ в силу однократного посещения каждой из вершин и не более чем двукратного
прохода по кажому из ребер (при спуске и подъеме).

Построение гамильтонова цикла на третьем шаге работает за $\mathcal{O}(|V|)$, так как мы просто один раз проходимся по списку вершин.

Итоговая асимптотика: $\mathcal{O}(|E| + |V| \log(|V|))$.

\subsection{Идея улучшения}
Для улучшения приближения с 2 до 1.5 нам нужно использовать алгоритм \textit{сжатия соцветий}~\cite{shoemaker2016blossom}.
Этот алгоритм позволяет найти совершенное паросочетание за $\mathcal{O}(|E||V|^2)$. Псевдокод алгоритма представлен ниже.

\begin{algorithm}[H]
    \caption{Поиск увеличивающего пути}
    \textbf{Вход:} Граф $G$, паросочетание $M$ в $G$.
    \begin{algorithmic}[1]
        \State $F \gets$ пустой лес
        \State Сделать все вершины и рёбра непомеченными в $G$, пометить все рёбра $M$
        \For{каждой голой вершины $v$}
        \State Создаём дерево из одной вершины $\{v\}$ и добавляем его в $F$
        \EndFor
        \While{имеется непомеченная вершина $v$ в $F$ с чётным расстоянием $distance(v, root(v))$}
        \While{существует непомеченное ребро $e = \{v, w\}$}
        \If{$w$ не в $F$}
        \State $x \gets$ сочетается с вершиной $w$ в $M$
        \State Добавляем рёбра $\{v, w\}$ и $\{w, x\}$ в дерево для $v$
        \ElsIf{$distance(w, root(w))$ нечётно}
        \State \textbf{continue}
        \Else
        \If{$root(v) \neq root(w)$}
        \State $P \gets$ путь $(root(v) \to \dots \to v) \to (w \to \dots \to root(w))$
        \State \Return $P$
        \Else
        \State $B \gets$ цветок, образованный $e$ и рёбрами пути $v \to w$ в $T$
        \State $(G', M') \gets$ сжать $G$ и $M$ путём стягивания цветка $B$
        \State $P' \gets$ \texttt{Найти увеличивающий путь}($G', M'$)
        \State $P \gets$ поднимаем $P'$ в $G$
        \State \Return $P$
        \EndIf
        \EndIf
        \State Помечаем ребро $e$
        \EndWhile
        \State Помечаем вершину $v$
        \EndWhile
        \State \Return пустой путь
    \end{algorithmic}
    \textbf{Выход:} Увеличивающий путь $P$ в $G$ или пустой путь, если такого пути не найдено.
\end{algorithm}

\begin{algorithm}[H]
    \caption{Нахождение наибольшего паросочетания}
    \textbf{Вход:} Граф $G$, начальное паросочетание $M$ на $G$, пустое на первой итерации.
    \begin{algorithmic}[1]
        \State $P \gets$ найти увеличивающий путь($G$, $M$)
        \If{$P$ не пустое}
        \State \Return \texttt{Найти наибольшее паросочетание}($G$, увеличиваем $M$ вдоль $P$)
        \Else
        \State \Return $M$
        \EndIf
    \end{algorithmic}
    \textbf{Выход:} Наибольшее паросочетание $M^*(G)$ на $G$.
\end{algorithm}


Также нам потребуется алгоритм поиска эйлерова пути в графе. Его сложность равна $\mathcal{O}(|E|)$. Псевдокод:

\begin{algorithm}[H]
    \caption{Алгоритм Эйлера}
    \begin{algorithmic}[1]
        \While{$g[v]$ не пуста}
        \State $u \gets \text{min}(g[v])$
        \State \texttt{remove\_edge}(v, u)
        \State \texttt{Эйлер}(u)
        \EndWhile
        \State \texttt{cout} $v$
    \end{algorithmic}
\end{algorithm}

\subsection{1.5-приближение}
Улучшим алгоритм 2-приближения так, чтобы получить требуемое.
Заметим, что константа $\alpha = 2$ получалась из-за того, что мы дважды проходились по всем ребрам.
Решим эту проблему.

Построим 1.5-приближение метрической задачи коммивояжера для графа $G = (V, E, w)$, воспользовавшись следующим алгоритмом:
\begin{enumerate}[label=\arabic*.]
    \item Построим $O^*(G)$ за полиномиальное время при помощи aлгоритма \textit{Прима}.
    \item Выберем в $O^*(G)$ вершины нечетной степени $V'$, получив индуцированный граф $O' = (V', E|_{V'})$ и найдем $M^*(O')$ при помощи алгоритма \textit{сжатия соцветий}~\cite{shoemaker2016blossom} за полиномиальное время.
    \item Добавим ребра из $M^*(O')$ в $O^*(G)$ с учетом кратности (добавляя даже если ребро уже существует), получив граф $O''$ за полиномиальное время.
    \item Найдем эйлеров цикл $E$ в $O''$ за полиномиальное время.
    \item Построим гамильтонов цикл $H'$, проходящий по вершинам в порядке их первого появления в $E$, так же за полиномиальное время,
          не забыв в конец добавить первую вершину для получения цикла.
\end{enumerate}

\begin{lemma}
    \label{lemma:2-3}
    $M^*(O')$ существует.
\end{lemma}
\begin{proof}
    Граф $O'$ будет полным с четным числом вершин (в силу четности количества вершин нечетной степени в любом графе, в частности в $O^*(G)$),
    а значит в нем существует полное паросочетание.
\end{proof}

\begin{lemma}
    \label{lemma:2-4}
    $E$ существует.
\end{lemma}
\begin{proof}
    $O'$ связен по определению.
    После добавления ребер из $M^*(O')$ мы увеличили степени всех вершин нечетной степени на 1.
    Следовательно, в $O'$ все вершины четные, а значит по критерию эйлеровости графа $O'$ является таковым и в нем есть эйлеров цикл.
\end{proof}

\begin{lemma}
    \label{lemma:2-5}
    Тогда $W(M^*(O')) \leq \frac12 W(H^*(G))$
\end{lemma}
\begin{proof}
    Построим гамильтонов цикл $H''$, посетив все вершины из $O'$ в том порядке, в котором они были в $H^*(G)$.
    По лемме~\ref{lemma:2-1}: $W(H'') \leq W(H^*(G))$.
    Теперь занумеруем все ребра в $H''$ в порядке их появления в цикле.
    Тогда ребра с четными и нечетными номерами поотдельности образуют полные паросочетания $M'$ и $M''$ в $O'$, причем $W(M') + W(M'') = W(H'')$.
    Тогда:
    $$W(M^*(O')) \leq \min\{M', M''\} \leq \frac12 W(H'') \leq \frac12 W(H^*(G))$$
\end{proof}

\begin{theorem}
    \label{theorem:2-2}
    Вышеприведенный алгоритм построения 1.5-приближения метрической задачи коммивояжера корректен.
\end{theorem}
\begin{proof}
    Достаточно показать, что $W(H') \leq \frac12 W(H^*(G))$. Воспользуемся полученными результатами.

    В силу леммы~\ref{lemma:2-1}: $W(H') \leq W(E)$.

    В силу леммы~\ref{lemma:2-2}: $W(O^*(G)) \leq W(H^*(G))$.

    В силу леммы~\ref{lemma:2-3}: $M^*(O')$ существует.

    В силу леммы~\ref{lemma:2-4}: $E$ существует.

    В силу леммы~\ref{lemma:2-5}: $W(M^*(O')) \leq \frac12 W(H^*(G))$.

    Получаем:
    $$W(H') \leq W(E) = W(O^*(G)) + W(M^*(O')) \leq W(H^*(G)) + \frac12 W(H^*(G)) \leq \frac12 W(H^*(G)),$$
    что и требовалось.
\end{proof}

\subsection*{Время работы}

Предложенный алгоритм 1.5-приближения работает в худшем случае за $\mathcal{O}(|E||V|^2)$,
эту оценку сверху дает алгоритм \textit{сжатия соцветий}~\cite{shoemaker2016blossom},
в то премя как алгоритм \textit{Прима} работает за $\mathcal{O}(|E| + |V| \log (|V|))$ и
поиск эйлерова цикла за $\mathcal{O}(|E| + |V|)$.

Не самое лучшее время работы, но зато мы смогли сильно улучшить верхнюю границу по сравнению с 2-приближением.
