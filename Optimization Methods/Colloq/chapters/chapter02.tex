\section{Билет 2}

\subsection{Удос. Формулировка условия локального минимума на \texorpdfstring{$\mathbb{R}^d$}{R} для произвольной
    непрерывно дифференцируемой функции. Формулировка условия глобального минимума на
    \texorpdfstring{$\mathbb{R}^d$}{R} для выпуклой непрерывно дифференцируемой
    функции. Формулировка условия глобального минимума на выпуклом
    множестве \texorpdfstring{$X$}{X} для выпуклой непрерывно дифференцируемой функции.}

\subsection{Хо-хо-хор. Доказательство условия локального минимума на \texorpdfstring{$R^d$}{R}.
    Доказательство условия глобального минимума на \texorpdfstring{$R^d$}{R} выпуклой функции.
    Доказательство условия глобального минимума на выпуклом множестве \texorpdfstring{$X$}{X} выпуклой функции.
    Доказательство свойства гладкой непрерывно дифференцируемой функции.
    Доказательство свойства ограниченности субградиента выпуклой Липшицевой функции.}


\begin{theorem}
    Пусть $x^*$ — локальный минимум функции $f$ на $\mathbb{R}^d$, тогда если $f$ дифференцируема, то $$\nabla f(x^*) = 0.$$
\end{theorem}
\begin{proof}
    Пойдем от противного и предположим, что $x^*$ — локальный минимум, но $\nabla f(x^*) \neq 0$.
    Разложим функцию $f$ в ряд в окрестности локального минимума:
    $$f(x) = f(x^*) + \langle \nabla f(x^*), x - x^* \rangle + o(\|x - x^*\|_2),$$
    где $$\lim_{x \to x^*} \frac{o(\|x - x^*\|_2)}{\|x - x^*\|_2} = 0.$$

    Рассмотрим $x_\lambda = x^* - \lambda \nabla f(x^*)$. Найдем $\lambda_1 > 0$ такое,
    что для любого $0 < \lambda \leq \lambda_1$ можно гарантировать,
    что $\|x_\lambda - x^*\|_2 \leq r$, т.е. $x_\lambda$ попадает в нужную окрестность из определения локального минимума.

    Понятно, что такое $\lambda_1$ можно найти в силу $r > 0$, а $\nabla f(x^*)$ конечно.
    Тогда для любого $0 < \lambda \leq \lambda_1$ справедливо $f(x_\lambda) \geq f(x^*)$.

    При этом разложение в ряд для точек $\tilde{x}_\lambda$ имеет вид:
    $$f(x_\lambda) = f(x^*) + \langle \nabla f(x^*), x_\lambda - x^* \rangle + o(\|x_\lambda - x^*\|_2)
        = f(x^*) - \lambda \|\nabla f(x^*)\|_2^2 + o(\lambda \|\nabla f(x^*)\|_2).$$

    Набросим еще одно ограничение на «малость» $\lambda$. А именно, найдем $\lambda_2 > 0$ такое,
    что для любого $0 < \lambda \leq \min\{\lambda_1, \lambda_2\}$ выполнено, что
    $$|o(\lambda \|\nabla f(x^*)\|_2)| \leq \frac{\lambda}{2} \|\nabla f(x^*)\|_2^2.$$

    Тогда для любого $\lambda > 0$, такого что $\lambda \leq \min\{\lambda_1, \lambda_2\}$, следует
    $$f(x_\lambda) \leq f(x^*) - \frac{\lambda^2}{2} \|\nabla f(x^*)\|^2.$$

    Пришли к противоречию, что $x^*$ — локальный минимум. А значит, $\nabla f(x^*) = 0$.
\end{proof}

\begin{theorem}
    Пусть дана выпуклая непрерывно дифференцируемая на $\mathbb{R}^d$ функция $f : \mathbb{R}^d \to \mathbb{R}$.
    Если для некоторой точки $x^* \in \mathbb{R}^d$ верно, что $\nabla f(x^*) = 0$,
    то $x^*$ — глобальный минимум $f$ на всем $\mathbb{R}^d$.
\end{theorem}
\begin{proof}
    Достаточно записать определение выпуклости:
    $$f(x) \geq f(x^*) + \langle \nabla f(x^*), x - x^* \rangle = f(x^*).$$
\end{proof}

\begin{theorem}
    Пусть дана выпуклая непрерывно дифференцируемая на $\mathbb{R}^d$ функция $f : \mathbb{R}^d \to \mathbb{R}$ и
    выпуклое множество $X$. Тогда $x^* \in X$ — глобальный минимум $f$ на $X$ тогда и только тогда,
    когда для всех $x \in X$ выполнено
    $$\langle \nabla f(x^*), x - x^* \rangle \geq 0.$$
\end{theorem}
\begin{proof}
    Доказательство в две стороны.

    \textbullet \, Достаточное условие. 
    Пусть $\langle \nabla f(x^*), x - x^* \rangle \geq 0$ для $x \in X$, тогда воспользуемся определением выпуклости:
    $$f(x) \geq f(x^*) + \langle \nabla f(x^*), x - x^* \rangle \geq f(x^*).$$
    Откуда следует, что $x^*$ — глобальный минимум на $X$ (Определение 6.2).

    \textbullet \, Необходимое условие. 
    Пусть $x^*$ — глобальный минимум на $X$. 
    Будем доказывать, что $\langle \nabla f(x^*), x - x^* \rangle \geq 0$ для любого $x \in X$. 
    Пойдем от противного, т.е. предположим, что существует $x \in X$ такой, что
    $$\langle \nabla f(x^*), x - x^* \rangle < 0.$$

    Рассмотрим точки вида
    $$x_\lambda = \lambda x + (1 - \lambda) x^*, \quad \lambda \in [0; 1].$$
    В силу выпуклости множества $X$ точки $x_\lambda \in X$. 

    Посмотрим, как ведет себя функция
    $$\varphi(\lambda) = f(x_\lambda) = f(\lambda x + (1 - \lambda) x^*).$$
    В частности, заметим, что
    $$\frac{d\varphi}{d\lambda} = \frac{d}{d\lambda} f(\lambda x + (1 - \lambda) x^*) = \langle \nabla f(x^* + \lambda(x - x^*)), x - x^* \rangle.$$

    Заметим, что
    $$\left. \frac{d\varphi}{d\lambda} \right|_{\lambda = 0} = \langle \nabla f(x^*), x - x^* \rangle < 0.$$
    
    Это значит, что функция $\varphi$ убывает в окрестности нуля. А значит, для достаточно малых $\lambda > 0$ выполнено
    $$f(x^* + \lambda (x - x^*)) = \varphi(\lambda) < \varphi(0) = f(x^*).$$
    Пришли к противоречию, что $x^*$ — глобальный минимум на $X$.
\end{proof}
