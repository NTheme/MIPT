\section{Блиет 9}

\subsection{Удос. Итерация метода зеркального спуска. Интуиция метода. Характер сходимости для выпуклых гладких задач.}

\begin{definition}
    Пространство $E^*$, которому принадлежит $\nabla f$ в градиентном спуске называется \guillemetleft зеркальным\guillemetright \,. Модифицируем:
    $$\phi(x_{k+1}) = \phi(x_k) - \gamma \nabla f(x_k)$$
\end{definition}
где $\phi$ подбирается так, что $\phi$ действует из $E$ в $E^*$, а $\phi^{-1}$ из $E^*$ в $E$.

В зеркальном пространстве делаем шаг градиентного спуска,
после чего с помощью $\phi^{-1}$ возвращаемся к $x_{k+1}$ из $E$.

\begin{definition}
    \textbf{µ-сильная выпуклость.} Пусть дана непрерывно дифференцируемая на выпуклом множестве \( X \) функция
    \( d : X \to \mathbb{R} \). Будем говорить, что она является µ-сильно выпуклой (\( \mu > 0 \)) относительно нормы \( \| \cdot \| \) на множестве \( X \), если для любых \( x, y \in X \) выполнено:
    $$d(x) \geq d(y) + \langle \nabla d(y), x - y \rangle + \frac{\mu}{2} \| x - y \|^2.$$
\end{definition}

\begin{definition}
    Для функции $d$, дифференцируемой и 1-сильно выпуклой относительно нормы $\| \cdot \|$ на множестве $X$,
    \textbf{дивергенцией Брэгмана}, порожденной функцией $d$ на множестве $X$,
    называется функция двух аргументов $V(x, y) : X \times X \to \mathbb{R}$ такая, что для любых $x, y \in X$ выполнено:
    $$V(x, y) = d(x) - d(y) - \langle \nabla d(y), x - y \rangle.$$
\end{definition}

\subsection*{Метод зеркального спуска:}
$$x_{k+1} = \arg \min_{x \in X} \left\{ \langle \gamma \nabla f(x_k), x \rangle + V(x, x_k) \right\}.$$

\subsection{Хорош! Шаг зеркального спуска в случае
    симплекса и KL-дивергенции (с доказательством).}

\begin{theorem}
    Пусть задача оптимизации на выпуклом множестве $X$ с $L$-гладкой относительно нормы $\| \cdot \|$,
    выпуклой целевой функцией $f$ решается с помощью зеркального спуска с шагом $\gamma \leq \frac{1}{L}$.
    Тогда справедлива следующая оценка сходимости:
    $$f\left( \frac{1}{K} \sum_{k=1}^K x_k \right) - f(x^*) \leq \frac{V(x^*, x_0)}{\gamma K}.$$
\end{theorem}
