\documentclass[a5paper,twoside,russian]{article}
\usepackage[intlimits]{amsmath}
\usepackage{amsthm,amsfonts}
\usepackage{amssymb}
\usepackage{mathrsfs}
\usepackage[final]{graphicx,epsfig}
\usepackage{indentfirst}
\usepackage[utf8]{inputenc}
\usepackage[T2A]{fontenc}
\usepackage[english]{babel}
\usepackage[usenames]{color}
\usepackage{hyperref}
\usepackage{wasysym}
\usepackage{enumitem}
\RequirePackage{enumitem}
\renewcommand{\alph}[1]{\asbuk{#1}}
\setenumerate[1]{label=\alph*), fullwidth, itemindent=\parindent, listparindent=\parindent} 
\setenumerate[2]{label=\arabic*), fullwidth, itemindent=\parindent, listparindent=\parindent, leftmargin=\parindent}
\usepackage{mathtools}

\usepackage{rmathbr}  % для автопереносов
\usepackage{setspace} % увеличение межстрочного расстояния

\usepackage{thmtools}
\renewcommand{\qed}{$\hfill\blacksquare$}
\declaretheorem{definition} 
\declaretheoremstyle[%
  spaceabove=-6pt,%
  spacebelow=6pt,%
  headfont=\normalfont\itshape,%
  postheadspace=1em,%
  qed=\qedsymbol,%
  headpunct={}
]{mystyle} 
\declaretheorem[name={$\blacktriangle$},style=mystyle,unnumbered,
]{prf}


\hoffset=-10.4mm \voffset=-12.4mm \oddsidemargin=5mm \evensidemargin=0mm \topmargin=0mm \headheight=0mm \headsep=0mm
\textheight=174mm \textwidth=113mm

\newcommand{\bfv}{\mathbf}

\def \e {\varepsilon}

\def \ZZ {\mathbb Z}
\def \FF {\mathbb F}
\def  \R {\mathbb R}
\def \QQ {\mathbb Q}
\def \NN {\mathbb N}
\def \PP {\mathbb P}
\def \EE {\mathbb E}
\def \DD {\mathbb D}
\def \CC {\mathbb C}
\def \II {\mathbb I}


\def\Ss{\mathcal{S}^n} 
\def\Pp{\mathcal{P}^{n}}
\def\cA{{\cal A}}
\def\cB{{\cal B}}
\def\cD{{\cal D}}
\def\cC{{\cal C}}
\def\cQ{{\cal Q}}
\def\cM{{\cal M}}
\def\cN{{\cal N}}
\def\cT{{\cal T}}
\def\cP{{\cal P}}
\def\cF{{\cal F}}
\newcommand{\St}{\mathbb{S}}
\def\la{\langle}
\def\ra{\rangle}

\def\tr{\text{Tr}}

\DeclareMathOperator*{\argmax}{arg\,max}
\DeclareMathOperator*{\argmin}{arg\,min}
\DeclareMathOperator*{\dom}{dom}

\newcommand{\Tr}{\operatorname{Tr}}
% Блочная матрица

\renewcommand{\Re}{\mathrm{Re}\,}
\renewcommand{\Im}{\mathrm{Im}\,}
\def\<{\langle}
\def\>{\rangle}

\begin{document}
\selectlanguage{russian}

\section{Домашнее задание 1, основная часть}

\begin{center}
    \textbf{Deadline - 20.09.2024 в 23:59}
\end{center}

В этой части используется следующие обозначения:

$\mathbb{R}_{++}$ - положительные вещественные числа

$I_n$ - матрица с единицами на диагонали (вне диагонали 0)

$A \in \mathbb{S}^n \quad\Longleftrightarrow \quad A= A^\top$

$A \in \mathbb{S}^n_+ \quad\Longleftrightarrow \quad A \in \mathbb{S}^n ; \quad  \forall x: \quad  x^\top Ax \geq 0$

$A \in \mathbb{S}^n_{++}\quad \Longleftrightarrow \quad A \in \mathbb{S}^n ; \quad \forall x \neq 0: \quad  x^\top Ax > 0$

Норма Фробениуса для матрицы $A \in \mathbb{R}^{n \times n}$ определяется как $||A||_F = \sqrt{\sum_{i=1}^n \sum_{j=1}^n A^2_{ij}}$

Для матриц скалярное произведение определено как $\la X, Y \ra := \text{Tr}(X^\top Y)$

\begin{enumerate}[label=\textbf{Задача \arabic*.}]

    \item Пусть $f$ -- одна из следующих функций:
    \begin{enumerate}
        \item (1 балл) $f : E \to \R$ -- функция $f(t) := \det(A - t I_n)$, где $A \in \R^{n \times n}$, $E := \{ t \in \R : \det(A - t I_n) \neq 0 \}$.
        
        \vspace{0.15cm}
        \begin{prf}
            $\hookrightarrow df = d \bigl( \det(A - t I_n) \bigr) = 
            \det(A - t I_n) \cdot \Tr \bigl( (A - t I_n)^{-1} \cdot d(A - t I_n) \bigr) = 
            -\det(A - t I_n) \cdot \Tr \bigl( (A - t I_n)^{-1} \bigr) dt \Rightarrow$

            \vspace{-0.1cm}
            \begin{center}
                $\Rightarrow \boxed{f'(t) = -\det(A - t I_n) \cdot \Tr\bigl((A - t I_n)^{-1}\bigr)}$
            \end{center}

            $\hookrightarrow d^2f = df' = d \Bigl( -\det(A - t I_n) \cdot \Tr\bigl((A - t I_n)^{-1}\bigr) \Bigr) = 
            -d \bigl( \det(A - t I_n) \bigr) \cdot \Tr \bigl((A - t I_n)^{-1} \bigr) - \det(A - t I_n) \cdot d \Bigl( \Tr\bigl((A - t I_n)^{-1}\bigr) \Bigr) = 
            \det(A - t I_n) \cdot \Tr^2 \bigl( (A - t I_n)^{-1} \bigr) dt - \det(A - t I_n) \cdot \Tr \Bigl( d \bigl( (A - t I_n)^{-1} \bigr) \Bigr) = 
            \det(A - t I_n) \cdot \Tr^2 \bigl( (A - t I_n)^{-1} \bigr) dt - \det(A - t I_n) \cdot \Tr \bigl( (A - t I_n)^{-1} \cdot (I_n dt) \cdot (A - t I_n)^{-1} \bigr) = 
            \det(A - t I_n) \cdot \Bigl( \Tr^2 \bigl( (A - t I_n)^{-1} \bigr) - \Tr \bigl( (A - t I_n)^{-2} \bigr) \Bigr) dt \Rightarrow$

            \vspace{-0.1cm}
            \begin{center}
                $\Rightarrow \boxed{f''(t) = \det(A - t I_n) \cdot \Bigl( \Tr^2 \bigl( (A - t I_n)^{-1} \bigr) - \Tr \bigl( (A - t I_n)^{-2} \bigr) \Bigr)}$
            \end{center}
        \end{prf}

        \item (1.5 балла) $f : \R_{++} \to \R$ -- функция $f(t) := \| (A + t I_n)^{-1} b \|^2$, где $A \in \St^n_{++}$, $b \in \R^n$.
        
        \vspace{0.15cm}
        \begin{prf}
            $\hookrightarrow df = d \| (A + t I_n)^{-1} b \|^2 = 
            2 \cdot \Bigl\la (A + t I_n)^{-1} b, d \bigl( (A + t I_n)^{-1} b \bigr) \Bigr\ra = 
            -2 \cdot \bigl\la (A + t I_n)^{-1} b, (A + t I_n)^{-2} b \cdot dt \bigr\ra = 
            -2 \Tr \bigl( b^T (A + t I_n)^{-T} (A + t I_n)^{-2} b dt \bigr) = 
            \{ \text{Так как } A, I_n \in \St^n \} = 
            -2 \Tr \bigl( b^T (A + t I_n)^{-3} b \bigr) \cdot dt = 
            -2 \cdot \bigl\la b, (A + t I_n)^{-3} b \bigr\ra \cdot dt \Rightarrow$

            \vspace{-0.1cm}
            \begin{center}
                $\Rightarrow \boxed{f'(t) = -2 \cdot \bigl\la b, (A + t I_n)^{-3} b \bigr\ra}$
            \end{center}

            $\hookrightarrow d^2f = df' = -2 \cdot \Bigl\la 0, (A + t I_n)^{-3} b \Bigr\ra - 2 \cdot \Bigl\la b, d \bigl( (A + t I_n)^{-3} \bigr) b \Bigr\ra = 
            -2 \cdot \Bigl\la b, -(A + t I_n)^{-3} \cdot d \bigl( (A + t I_n)^{3} \bigr) \cdot (A + t I_n)^{-3} b \Bigr\ra = 
            6 \cdot \Bigl\la b, (A + t I_n)^{-3} \cdot (A + t I_n)^{2} \cdot d \bigl( (A + t I_n) \bigr) \cdot (A + t I_n)^{-3} b \Bigr\ra = 
            6 \cdot \bigl\la b, (A + t I_n)^{-4} b \bigr\ra \cdot dt \Rightarrow$

            \vspace{-0.1cm}
            \begin{center}
                $\Rightarrow \boxed{f''(t) = 6 \cdot \bigl\la b, (A + t I_n)^{-4} b \bigr\ra}$
            \end{center}
        \end{prf}

    \end{enumerate}
    Для каждого из указанных вариантов вычислите первую и вторую производные $f'(t)$ и $f''(t)$.

    \item Пусть $f$ -- одна из следующих функций:
    \begin{enumerate}
        \item (2 балла) $f : \R^n \to \R$ -- функция $\displaystyle f(x) := \frac{1}{2} \| x x^T - A \|_F^2$, где $A \in \St^n$.
        
        \vspace{0.15cm}
        \begin{prf}
            $\hookrightarrow df = d(\frac{1}{2} \| x x^T - A \|_F^2) = 
            \frac{1}{2} d \Bigl( \Tr \bigl( (x x^T - A)^T (x x^T - A) \bigr) \Bigr) = 
            \frac{1}{2} \Tr \Bigl( d \bigl( (x x^T - A)^T (x x^T - A) \bigr) \Bigr) = 
            \frac{1}{2} \Tr \Bigl( d \bigl( (x x^T - A)^T \bigr) (x x^T - A) + (x x^T - A)^T d (x x^T - A) \Bigr) = 
            \{ \text{Так как } (x x^T - A) \in \St^n \} = 
            \Tr \bigl( (x x^T - A)^T d (x x^T - A) \bigr) = 
            \Tr \Bigl( (x x^T - A)^T \bigl( dx x^T + x d (x^T) \bigr) \Bigr) = 
            \Tr \bigl( (x x^T - A)^T dx x^T \bigr) + \Tr \bigl( (x x^T - A)^T x d (x^T) \bigr) = 
            \Tr \bigl( x^T (x x^T - A)^T dx \bigr) + \Tr \bigl( x^T (x x^T - A) dx \bigr) = 
            \Tr \bigl( 2 x^T (x x^T - A)^T dx \bigr) =
            \bigl\la 2 (x x^T - A) x, dx \bigr\ra \Rightarrow$

            \vspace{-0.1cm}
            \begin{center}
                $\Rightarrow \boxed{\nabla f(x) = 2 (x x^T - A) x}$
            \end{center}

            $\hookrightarrow d(\nabla f) = d \bigl( 2 (x x^T - A) x \bigr) = 
            2 \bigl( d (x x^T x) - A dx \bigr) = 
            2 \Bigl( d \bigl( x \cdot \la x, x \ra \bigr) - A dx \Bigr) = 
            2 \Bigl( dx \cdot \la x, x \ra + x d \bigl( \la x, x \ra \bigr) - A dx \Bigr) = 
            2 \bigl( \la x, x \ra \cdot dx + x \cdot \la 2x, dx \ra - A dx \bigr) = 
            \{ \text{Так как } (2 x^T x I_n + 4 x x^T - 2 A) \in \St^n \} =
            (2 x^T x I_n + 4 x x^T - 2 A)^T dx \Rightarrow$

            \vspace{-0.1cm}
            \begin{center}
                $\Rightarrow \boxed{\nabla^2 f(x) = 2 x^T x I_n + 4 x x^T - 2 A}$
            \end{center}
        \end{prf}

        \item (2.5 балла) $f : \R^n \setminus \{ 0 \} \to \R$ -- функция $\displaystyle f(x) := \la x, x \ra^{\la x, x \ra}$.
        
        \vspace{0.15cm}
        \begin{prf}
            $\hookrightarrow df = d \bigl( \la x, x \ra^{\la x, x \ra} \bigr) = 
            d e^{\la x, x \ra \cdot \ln \bigl( \la x, x \ra \bigr)} = 
            e^{\la x, x \ra \cdot \ln \bigl( \la x, x \ra \bigr)} \cdot d \Bigl( \la x, x \ra \cdot \ln \bigl( \la x, x \ra \bigr) \Bigr) = 
            \la x, x \ra^{\la x, x \ra} \cdot \Bigl( d \bigl( \la x, x \ra \bigr) \cdot \ln \bigl( \la x, x \ra \bigr) + d \bigl( \la x, x \ra \bigr) \Bigr) = 
            \la x, x \ra^{\la x, x \ra} \cdot \Bigl( 1 + \ln \bigl( \la x, x \ra \bigr) \Bigr) \cdot d \bigl( \la x, x \ra \bigr) = 
            2 \cdot \la x, x \ra^{\la x, x \ra} \cdot \Bigl( 1 + \ln \bigl( \la x, x \ra \bigr) \Bigr) \cdot \la x, dx \ra = 
            \bigl\la 2 \cdot \la x, x \ra^{\la x, x \ra} \cdot \Bigl( 1 + \ln \bigl(\la x, x \ra \bigr) \Bigr) \cdot x, dx \bigr\ra \Rightarrow$

            \vspace{-0.1cm}
            \begin{center}
                $\Rightarrow \boxed{\nabla f(x) = 2 \cdot \la x, x \ra^{\la x, x \ra} \cdot \Bigl( 1 + \ln \bigl( \la x, x \ra \bigr) \Bigr) \cdot x}$
            \end{center}

            $\hookrightarrow d(\nabla f) = d \biggl( 2 \cdot \la x, x \ra^{\la x, x \ra} \cdot \Bigl( 1 + \ln \bigl( \la x, x \ra \bigr) \Bigr) \cdot x \biggr) = 
            2 \cdot d \bigl( \la x, x \ra^{\la x, x \ra} \bigr) \cdot \Bigl( 1 + \ln \bigl( \la x, x \ra \bigr) \Bigr) \cdot x + 
            2 \cdot \la x, x \ra^{\la x, x \ra} \cdot d \Bigl( 1 + \ln \bigl( \la x, x \ra \bigr) \Bigr) \cdot x +
            2 \cdot \la x, x \ra^{\la x, x \ra} \cdot \Bigl( 1 + \ln \bigl( \la x, x \ra \bigr) \Bigr) \cdot dx = 
            4 \cdot \la x, x \ra^{\la x, x \ra} \cdot \Bigl( 1 + \ln \bigl( \la x, x \ra \bigr) \Bigr)^2 \cdot \la x, dx \ra \cdot x + 
            4 \cdot \la x, x \ra^{\la x, x \ra - 1} \cdot \la x, dx \ra \cdot x + 
            2 \cdot \la x, x \ra^{\la x, x \ra} \cdot \Bigl( 1 + \ln \bigl( \la x, x \ra \bigr) \Bigr) \cdot dx = 
            4 \cdot \la x, x \ra^{\la x, x \ra} \cdot \Bigl( 1 + \ln \bigl( \la x, x \ra \bigr) \Bigr)^2 x \cdot \la x, dx \ra + 
            4 \cdot \la x, x \ra^{\la x, x \ra - 1} \cdot x \cdot \la x, dx \ra + 
            2 \cdot \la x, x \ra^{\la x, x \ra} \cdot \Bigl( 1 + \ln \bigl( \la x, x \ra \bigr) \Bigr) dx =
            4 \cdot \la x, x \ra^{\la x, x \ra} \cdot \Bigl( 1 + \ln \bigl( \la x, x \ra \bigr) \Bigr)^2 x x^T dx + 
            4 \cdot \la x, x \ra^{\la x, x \ra - 1} \cdot x x^T dx + 
            2 \cdot \la x, x \ra^{\la x, x \ra} \cdot \Bigl( 1 + \ln \bigl( \la x, x \ra \bigr) \Bigr) dx$

            \begin{center}
                $\Rightarrow \boxed{
                \begin{aligned}
                    \nabla^2 f(x) = 4 \cdot \la x, x \ra^{\la x, x \ra} \cdot \Bigl( 1 + \ln \bigl( \la x, x \ra \bigr) \Bigr)^2 x x^T dx \text{ } + \\ 
                    + \text{ } 4 \cdot \la x, x \ra^{\la x, x \ra - 1} \cdot x x^T dx + 
                    2 \cdot \la x, x \ra^{\la x, x \ra} \cdot \Bigl( 1 + \ln \bigl( \la x, x \ra \bigr) \Bigr)
                \end{aligned}}$
            \end{center}
        \end{prf}

    \end{enumerate}
    Для каждого из указанных вариантов вычислите градиент $\nabla f$ и гессиан $\nabla^2 f$ (относительно стандартного скалярного произведения в пространстве $\R^n$).

    \item Для каждой из следующих функций $f$ покажите, что вторая производная $d^2 f$ является знакоопределенной (как квадратичная форма) и установите ее знак:
    \begin{enumerate}
        \item (3 балла) $f : \St^n_{++} \to \R$ -- функция $f(X) := \la X^{-1}, A \ra$, где $A \in \St^n_+$.
    \end{enumerate}
        
\end{enumerate}

\end{document}
