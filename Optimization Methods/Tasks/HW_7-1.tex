\documentclass[a5paper,twoside,russian,8pt]{article}
\usepackage[intlimits]{amsmath}
\usepackage{amsthm,amsfonts}
\usepackage{amssymb}
\usepackage{mathrsfs}
\usepackage[final]{graphicx,epsfig}
\usepackage{indentfirst}
\usepackage[utf8]{inputenc}
\usepackage[T2A]{fontenc}
\usepackage[english]{babel}
\usepackage[usenames]{color}
\usepackage{hyperref}
\usepackage{wasysym}
\usepackage{enumitem}
\RequirePackage{enumitem}
\renewcommand{\alph}[1]{\asbuk{#1}}
\setenumerate[1]{label=\alph*), fullwidth, itemindent=\parindent, listparindent=\parindent} 
\setenumerate[2]{label=\arabic*), fullwidth, itemindent=\parindent, listparindent=\parindent, leftmargin=\parindent}
\usepackage{mathtools}

\hoffset=-10.4mm \voffset=-12.4mm \oddsidemargin=5mm \evensidemargin=0mm \topmargin=0mm \headheight=0mm \headsep=0mm
\textheight=174mm \textwidth=113mm

\newcommand{\bfv}{\mathbf}

\def \e {\varepsilon}

\def \ZZ {\mathbb Z}
\def \FF {\mathbb F}
\def  \R {\mathbb R}
\def \QQ {\mathbb Q}
\def \NN {\mathbb N}
\def \PP {\mathbb P}
\def \EE {\mathbb E}
\def \DD {\mathbb D}
\def \CC {\mathbb C}
\def \II {\mathbb I}


\def\Ss{\mathcal{S}^n} 
\def\Pp{\mathcal{P}^{n}}
\def\cA{{\cal A}}
\def\cB{{\cal B}}
\def\cD{{\cal D}}
\def\cC{{\cal C}}
\def\cQ{{\cal Q}}
%\def\R{{\cal R}}
\def\cM{{\cal M}}
\def\cN{{\cal N}}
\def\cT{{\cal T}}
\def\cP{{\cal P}}
\def\cF{{\cal F}}
\newcommand{\St}{\mathbb{S}}
\def\la{\langle}
\def\ra{\rangle}

\DeclareMathOperator*{\argmax}{arg\,max}
\DeclareMathOperator*{\argmin}{arg\,min}
\DeclareMathOperator*{\dom}{dom}

\newcommand{\Tr}{\operatorname{Tr}}
% Блочная матрица

\renewcommand{\Re}{\mathrm{Re}\,}
\renewcommand{\Im}{\mathrm{Im}\,}
\def\<{\langle}
\def\>{\rangle}




\begin{document}
\selectlanguage{russian}
\begin{center}
    \textbf{Домашнее задание 7, сопряженные множества и функции}
\end{center}
\begin{center}
    \textbf{Deadline - 01.11.2024 в 23:59}
\end{center}

\section*{Основная часть}
\subsection*{Сопряженные множества}
\begin{enumerate}[label=\textbf{Задача \arabic*.}]

\item Постройте $X^*$, если
\begin{enumerate}

    \item (0.5 балла) $X = \{x \in \R^2 : x_1^2+(x_2-1)^2 \leq 1\}$
    \item (0.5 балла) $X = \{x \in \R^2 : x_2\geq \frac{x_1^2}{2}\}$
    \item (0.5 балла) $X = \{x \in \R^d : \|Ax\| \leq 1, A \in \St^d_{++} \}$
    
\end{enumerate}

\item (1 балл) Постройте $X^*$ и $X^{**}$, если 

$X = \left\{x \in \R^d : \sum\limits_{i=1}^d x_i = 1 \right\}$

\subsection*{Сопряженные функции}

\item Постройте $f^{*}(y)$, если
\begin{enumerate}
    \item (0.5 балла) $f(x) = \max(|x|, x^2), x\in \R$
    \item (0.5 балла) $f(x) = (|x| + 1)\ln(|x|+1), x\in \R$
    \item (0.5 балла) $f(x)=(\|x\|+1)\ln(\|x\|+1)-\|x\|, x\in \R^d$
    
\end{enumerate}

\item (1 балл) Постройте $f^*(y)$ и $f^{**}(x)$, если 

$f(x) = \ch x - 1 = \frac{e^x+e^{-x}}{2} - 1, x\in \R$

\end{enumerate}

\section*{Дополнительная часть}

\subsection*{Сопряженные множества}
\begin{enumerate}[label=\textbf{Задача \arabic*.}]
    \item (1 балл) Пусть $\Bbb{A}^d$ - множеством антисимметричных матриц, т.е. $X\in \Bbb{A}^d \Leftrightarrow X^T = -X$. Покажите, что $(\Bbb{A}^d)^*=\St^d$.
    
    \item Докажите, что $K_p$ и $K_{p_*}$ являются взаимосопряженными, т.е. $(K_p)^* = K_{p_*}$ и $(K_{p_*})^* = K_p$, где $K_p=\{(x,\mu)\in R^{d+1}:\|x\|_p\leq\mu\}$ и 
        \begin{enumerate}
            \item (0.75 балла) $1 < p < \infty,\ \frac{1}{p}+\frac{1}{p_*} = 1$
            \item (0.75 балла) $p=1,\ p_*=\infty$
        \end{enumerate}
    
    \subsection*{Сопряженные функции}
    
    \item (0.75 балла)Постройте $f^*(x)$, если $f(x) = \log \left(\sum_{i=1}^d e^{x_i} \right) $, $x\in \R^d$
    
    \item Постройте $f^*(x)$, если:
        \begin{enumerate}
            \item (0.5 балла) $f(x) = \alpha g(x)$, где $\alpha > 0$
            \item (0.5 балла) $f(x) = g(Ax)$, где $A\in \St^d_{++}$
            \item (0.75 балла) $f(x) = \inf\limits_{u+v=x}(g(u)+h(v))$
        \end{enumerate}

\end{enumerate}
\end{document}