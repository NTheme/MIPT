\documentclass[a5paper,twoside,russian]{article}
\usepackage[intlimits]{amsmath}
\usepackage{amsthm,amsfonts}
\usepackage{amssymb}
\usepackage{mathrsfs}
\usepackage[final]{graphicx,epsfig}
\usepackage{indentfirst}
\usepackage[utf8]{inputenc}
\usepackage[T2A]{fontenc}
\usepackage[english]{babel}
\usepackage[usenames]{color}
\usepackage{hyperref}
\usepackage{ulem}
\usepackage{bookmark}
\usepackage{tikz}
\usepackage{wasysym}
\usepackage{enumitem}
\renewcommand{\alph}[1]{\asbuk{#1}}
\setenumerate[1]{label=\alph*), fullwidth, itemindent=\parindent, listparindent=\parindent}
\setenumerate[2]{label=\arabic*), fullwidth, itemindent=\parindent, listparindent=\parindent, leftmargin=\parindent}
\usepackage{mathtools}

\usepackage{rmathbr}  % для автопереносов
\usepackage{setspace} % увеличение межстрочного расстояния

\usepackage{thmtools}
\renewcommand{\qed}{$\hfill\blacksquare$}
\declaretheorem{definition}
\declaretheoremstyle[%
    spaceabove=-6pt,%
    spacebelow=6pt,%
    headfont=\normalfont\itshape,%
    postheadspace=1em,%
    qed=\qedsymbol,%
    headpunct={}
]{myStyle}
\declaretheorem[name={$\blacktriangle$},style=myStyle,unnumbered,
]{prf}

\hoffset=-10.4mm \voffset=-12.4mm \oddsidemargin=5mm \evensidemargin=0mm \topmargin=0mm \headheight=0mm \headsep=0mm
\textheight=174mm \textwidth=113mm

\DeclareMathOperator*{\argmax}{arg\,max}
\DeclareMathOperator*{\argmin}{arg\,min}
\DeclareMathOperator*{\dom}{dom}
\newcommand{\Tr}{\operatorname{Tr}}

\begin{document}
    \selectlanguage{russian}
    \begin{center}
        \textbf{Домашнее задание 9, ККТ}
    \end{center}
    \begin{center}
        \textbf{Deadline - 15.11.2024 в 23:59}
    \end{center}

    \section*{Основная часть}

    \begin{enumerate}[label=\textbf{Задача \arabic*.}]

        \item (2 балла)

        \begin{align*}
            \min_{\theta, \theta_0, \xi} &~ \frac{1}{2}\|\theta\|_2^2 + \rho\sum\limits_{i = 1}^{n}\xi_i\\
            \text{s.t. } \xi_i \geq 0, &~ y_i(x_i^\top \theta + \theta_0) \geq 1 - \xi_i
        \end{align*}

        Выпишите условия ККТ для этой задачи.
        Выразите параметр $\theta$ через оптимальное решение двойственной задачи.
        Покажите, что оптимальное решение не изменится, если оставить только точки $x_i$,
        для которых $y_i(x_i^\top \theta + \theta_0) = 1 - \xi_i$ (\emph{опорные векторы}).

        \begin{prf}
            $ \mathcal{L} (\theta, \theta_0, \xi, \lambda, \eta) =
            \frac12 ||\theta||_2^2 + \rho \sum\limits_{i=1}^{n}\xi_i +
            \sum\limits_{i=1}^{n}\lambda_i \left(1 - \xi_i - y_i \left( x_i^T \theta + \theta_0 \right) \right) -
            \sum\limits_{i=1}^{n} \eta_i \xi_i $

            \[
                \text{KKT:}
                \left\{\!
                \begin{array}{l}
                    \frac{\partial}{\partial \theta} \mathcal{L} = \theta - \sum\limits_{i=1}^{n} \lambda_i y_i x_i = 0 \\
                    \frac{\partial}{\partial \theta_0} \mathcal{L} = -\sum\limits_{i=1}^{n} \lambda_i y_i = 0           \\
                    \frac{\partial}{\partial \xi_i} \mathcal{L} = \rho - \lambda_i + \eta_i = 0                         \\
                    \lambda_i \geq 0                                                                                    \\
                    \eta_i \geq 0                                                                                       \\
                    \xi_i \geq 0                                                                                        \\
                    y_i \left( x_i^T \theta + \theta_0 \right) - 1 + \xi_i \geq 0                                       \\
                    \lambda_i \left( y_i \left( x_i^T \theta + \theta_0 \right) - 1 + \xi_i \right) = 0                 \\
                    \eta_i\xi_i = 0                                                                                     \\
                \end{array}
                \!\right.
            \]
            Тогда: $ \theta = \sum\limits_{i=1}^{n} \lambda_i^* y_i x_i$, где $\lambda_i^*$ - оптимальное решение двойственной задачи.

            Если оставить только те условия, где $y_i(x_i^\top \theta + \theta_0) = 1 - \xi_i$, то по условиям дополняющей нежесткости
            для остальных $\lambda_i^* = 0$, а значит сумму не меняет, следовательно, можно их выкинуть.
        \end{prf}

        \item (3 балла) Пусть $f$ -- сильно выпуклая дифференцируемая на $\mathbb{R}^n$ функция,
        матрица $A \in \mathbb{R}^{n \times d}$ такова, что все ее миноры имеют полный ранг.


        Докажите, что задача с $\ell_1$ регуляризацией
        \begin{align*}
            \min_{x \in \mathbb{R}^d} \bigg[f(Ax) + \|x\|_1\bigg]
        \end{align*}
        имеет единственное решение, при чем среди его компонент не более $\min(n, d)$ ненулевых.

        \begin{prf}
            Так как $f(x)$ - сильно выпуклая, а $\ell_1$ - выпуклая, то сумма этих функций является сильно выпуклой функцией
            ($\ell_1$ не меняет константу выпуклости так как добавляет не менее 0 ко второму дифференциалу или не менее 0
            к первому в зависимости от того, какой критерий применить).
            Значит, решение задачи единственно в силу единственности минимума у строго выпуклой функции.

            Если $n \geq d$, то тривиально: не можем занулить переменных больше, чем их есть.
            
            Иначе у нас остаются свободные переменные и множество решений имеет размерность $d - n$, потому мы можем занулить
            часть переменных, не повлияв на значение минимума суммы.
            Значит количество ненулевых окажется равным $d$, то есть в любом случае не превосходит $\min(n, d)$.
        \end{prf}

    \end{enumerate}

\end{document}
