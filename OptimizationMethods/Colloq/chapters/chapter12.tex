\section{Билет 12}

\subsection{Удос. Штрафная функция. Метод штрафных функций. Постановка задачи и итерация метода ADMM.}

\subsection*{Плоти штраф}

\begin{equation}
    \begin{aligned}
         & \max_{x \in \mathbb{R}^d} f(x)              \\
         & \text{s.t. } h_i(x), \quad i = 1, \dots, m.
    \end{aligned}
\end{equation}

Возьмем некоторое $\rho > 0$ и немного модифицируем нашу задачу:
$$\min_{x \in \mathbb{R}^d} \left[ f(x) + \rho \cdot \frac{1}{2} \sum_{i=1}^{m} h_i^2(x) \right],$$
при этом $h_i(x) = 0$, $i = 1, \dots, m$.

\textbf{Замечание 20.1.} Можно заметить, что новая задача эквивалентна старой, так как для $x$,
удовлетворяющих ограничениям, \guillemotleft добавка\guillemetright \, равна 0.

Перепишем задачу в следующем виде:
$$\min_{x \in \mathbb{R}^d} \left[ f_{\rho}(x) = f(x) + \rho \cdot \frac{1}{2} \sum_{i=1}^{m} h_i^2(x) \right].$$

\subsection*{ADMM}

\begin{equation}
    \begin{aligned}
         & \max_{x \in \mathbb{R}^{d_x}, x \in \mathbb{R}^{d_y}} f(x) + g(x) \\
         & \text{s.t. } Ax + By = c,
    \end{aligned}
\end{equation}

Где $ A \in \mathbb{R}^{n \times d_x} $, $ B \in \mathbb{R}^{n \times d_y} $, $ c \in \mathbb{R}^{n} $.

Аугментация:
$$\min_{x \in \mathbb{R}^{d_x}, y \in \mathbb{R}^{d_y}} f(x) + g(y) + \frac{\rho}{2} \| Ax + By - c \|_2^2, \quad \text{s.t.} \quad Ax + By = c.$$

Лагранжиан:
$$L_{\rho}(x, y, \lambda) = f(x) + g(y) + \lambda^T (Ax + By - c) + \frac{\rho}{2} \| Ax + By - c \|_2^2.$$

\begin{note}
    Легко видеть, что такой Лагранжиан порождает выпукло-вогнутую седловую задачу.
\end{note}

\begin{algorithm}[H]
    \caption{ADMM}
    \textbf{Вход:} стартовая точка $x_0 \in \mathbb{R}^{d_x}$, $y_0 \in \mathbb{R}^{d_y}$, $\lambda_0 \in \mathbb{R}^n$, количество итераций $K$
    \begin{algorithmic}[1]
        \For{$k = 0, 1, \dots, K - 1$}
        \State $x_{k+1} = \arg \min_{x \in \mathbb{R}^{d_x}} L_{\rho}(x, y_k, \lambda_k)$
        \State $y_{k+1} = \arg \min_{y \in \mathbb{R}^{d_y}} L_{\rho}(x_{k+1}, y, \lambda_k)$
        \State $\lambda_{k+1} = \lambda_k + \rho (A x_{k+1} + B y_{k+1} - c)$
        \EndFor
    \end{algorithmic}
    \textbf{Выход:} $\frac{1}{K} \sum_{k=1}^{K} x_k, \quad \frac{1}{K} \sum_{k=1}^{K} y_k, \quad \frac{1}{K} \sum_{k=1}^{K} \lambda_k$
\end{algorithm}

\subsection{Хор. Формулировка свойств решения штрафной задачи.}

\begin{lemma}
    Пусть $x^*$ — решение исходной задачи, а $x^*_{\rho}$ — решение соответствующей штрафной задачи с $\rho > 0$, тогда
    $$f(x^*) \geq f(x^*_{\rho}).$$
\end{lemma}

\begin{lemma}
    С увеличением $\rho$ решения штрафной задачи при его существовании гарантировано не ухудшают степень нарушения ограничений, т.е. для $\rho_1 > \rho_2$ следует, что
    $$\sum_{i=1}^{m} h_i^2(x^*_{\rho_2}) \geq \sum_{i=1}^{m} h_i^2(x^*_{\rho_1}),$$
    где $x^*_{\rho_1}$ и $x^*_{\rho_2}$ — решения соответствующих штрафных задач.
\end{lemma}

\begin{lemma}
    Пусть функция $f$ и все функции $h_i$ ($i = 1, \dots, m$) являются непрерывными. Пусть $X^*$ — множество решений исходной условной задачи оптимизации, и для $x^* \in X^*$ множество
    $$U = \{x \in \mathbb{R}^d \mid f(x) \leq f(x^*)\}$$
    ограничено.
    Тогда для любого $e > 0$ существует $\rho(e) > 0$ такое, что множество решений штрафной задачи $X^*_\rho$ для любых $\rho \geq \rho(e)$ содержится в
    $$X^*_e = \{x \in \mathbb{R}^d \mid \exists x^* \in X^* : \|x - x^*\|_2 \leq e\}.$$
\end{lemma}
