\section{Семестр}

\subsection{В чем заключается смысл физической культуры как компонента культуры общества?}



\subsection{Перечислите 3 группы признаков, характеризующих сферу физической культуры.}



\subsection{Дайте сравнительную характеристику понятиям «физическая культура» и «физическое воспитание»; «физическая культура» и «спорт».}



\subsection{Что является результатом физической подготовки?}



\subsection{Какими показателями характеризуется процесс физического развития?}



\subsection{Перечислите законы, которыми определяется и которым подчиняется процесс физического развития человека.}



\subsection{Перечислите важнейшие показатели физически совершенного человека современности.}



\subsection{В чем смысл терминов «физическая рекреация» и «двигательная реабилитация»?}



\subsection{Какие критерии положены Л.П. Матвеевым в основу выделения тех или иных разделов (видов) физической культуры?}



\subsection{На основании каких признаков структурирует В.М. Выдрин физическую культуру?}



\subsection{Перечислите потребности общества в сфере физической культуры.}



\subsection{Какие потребности личности удовлетворяют те или иные компоненты (виды) физической культуры?}



\subsection{Назовите основные виды физической культуры.}



\subsection{Перечислите отличительные признаки базовой физической культуры.}



\subsection{Раскройте понятие оздоровительно-реабилитационной физической культуры.}



\subsection{На что направлено содержание фоновой физической культуры?}



\subsection{Какие две категории функций физической культуры вы знаете?}



\subsection{В чем заключаются общекультурные функции физической культуры?}



\subsection{Что понимают под специфическими функциями физической культуры?}



\subsection{Раскройте содержание образовательной функции.}



\subsection{К какой категории наук относится теория физической культуры?}



\subsection{Перечислите источники возникновения и развития теории и методики физической культуры.}



\subsection{Охарактеризуйте особенности «Теории и методики физической культуры» как учебной дисциплины.}
