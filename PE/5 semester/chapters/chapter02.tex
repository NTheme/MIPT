\section{Семестр}

\subsection{Какие области знаний интегрирует в себе теория физической культуры?}

\begin{enumerate}
    \item Физиология: Изучает функционирование организма во время физической активности
    \item Биомеханика: Исследует механические принципы движения человеческого тела, анализирует оптимальные техники выполнения различных физических упражнений
    \item Психология: Изучает влияние физической активности на психическое здоровье
    \item Социология: Анализирует социальные и культурные аспекты физической культуры, включая роль физической активности в обществе
    \item Педагогика: Разрабатывает методы преподавания и организации физического образования, включая планирование тренировок
    \item Медицина и здравоохранение: Изучает взаимосвязь между физической активностью и здоровьем, лечение
    \item История и философия физической культуры: Анализирует эволюцию и историю физической культуры
\end{enumerate}

\subsection{Раскройте значение теории и методики физической культуры для студентов и выпускников физкультурных учебных заведений.}

\begin{enumerate}
    \item Теоретические основы: Студенты и выпускники получают знания о физиологических, биомеханических и психологических аспектах физической активности. Это помогает им лучше понимать функционирование организма, принципы движения и влияние физической активности на психическое здоровье.
    \item Разработка программ тренировок и занятий: Студенты учатся анализировать потребности и цели участников, выбирать оптимальные упражнения и методы обучения, планировать прогрессию и оценивать результаты
    \item Педагогические навыки: Теория и методика физической культуры помогают студентам и выпускникам развить навыки преподавания и руководства в области физической активности и спорта. Они учатся эффективно коммуницировать, мотивировать студентов
    \item Понимание социального и культурного контекста: Студенты и выпускники понимают роль физической активности в обществе, различия в предпочтениях и потребностях разных групп людей
\end{enumerate}

\subsection{Что понимается под системой ``физической культуры''?}

Система физической культуры - это комплекс взаимосвязанных элементов, институций и практик, направленных на развитие физической активности и спорта в обществе. Она включает в себя образовательные программы, тренировки, спортивные мероприятия, инфраструктуру, здравоохранение и культурные практики, связанные с физической активностью.

\subsection{Какова цель системы физической культуры?}

Целью системы физической культуры является содействие здоровому образу жизни, физическому развитию, спортивным достижениям и общественной поддержке активного образа жизни.

\subsection{Что входит в понятие ``система физического воспитания''?}

\begin{enumerate}
    \item Образовательные программы: Разработка и реализация образовательных программ, которые включают физическую активность и спорт в учебный план
    \item Физические тренировки: Организация тренировочных занятий, направленных на развитие физических качеств, координации, силы, выносливости и гибкости
    \item Соревнования и спортивные мероприятия: Проведение соревнований и спортивных мероприятий, которые стимулируют развитие соревновательного духа
    \item Физическое здоровье и реабилитация: Внедрение программ, направленных на поддержание и улучшение физического здоровья
    \item Психологическая поддержка: Предоставление психологической поддержки и консультаций для поддержки психологического благополучия и развития личности через физическую активность.
\end{enumerate}

\subsection{Как взаимодействую между собой педагогическая и социальная подсистема физического воспитания?}

\begin{enumerate}
    \item Цели и задачи: Педагогическая подсистема определяет образовательные цели и задачи, связанные с физическим развитием. Социальная подсистема определяет цели и задачи, связанные с формированием здорового образа жизни, социализацией, интеграцией в общество
    \item Образовательные программы и политики: Педагогическая подсистема разрабатывает и реализует образовательные программы, методики преподавания, оценку прогресса и обучающую среду, чтобы обеспечить эффективное обучение и развитие студентов в области физической активности. Социальная подсистема формулирует социальные политики, регулирующие доступность, равенство возможностей и социальную поддержку физической активности для всех членов общества.
    \item Социальная среда и инфраструктура: Социальная подсистема создает социальную среду, которая поддерживает и поощряет физическую активность, спорт и здоровый образ жизни. Это может включать создание спортивных объектов, парков, зон для занятий физической культурой, а также организацию мероприятий и спортивных сообществ, которые способствуют социализации, взаимодействию и поддержке участников.
\end{enumerate}

\subsection{Чем определяется постановка целей систем физической культуры?}

\begin{enumerate}
    \item Здоровье и физическое благополучие: Одной из основных целей систем физической культуры является поддержание и улучшение здоровья и физического состояния людей
    \item Развитие и самосовершенствование: Системы физической культуры ставят перед собой задачу развития физических, психологических и социальных аспектов личности.
    \item Общественная поддержка и социальное влияние: Системы физической культуры также преследуют цели, связанные с общественной поддержкой и социальным влиянием.
\end{enumerate}

\subsection{В чем выразилось влияние социально-экономических и других преобразований в нашей стране на состояние физической культуры?}

\begin{enumerate}
    \item Инфраструктура и доступность: Преобразования могут повлиять на развитие спортивной инфраструктуры и доступность объектов для занятий физической культурой.
    \item Образование и тренировка кадров: Изменения в системе образования и тренировки кадров могут повлиять на качество и доступность профессионального обучения в области физической культуры
    \item Общественное сознание и стереотипы: Социальные и экономические изменения могут влиять на общественное сознание и стереотипы относительно физической культуры. Появление новых ценностей
\end{enumerate}

\subsection{Охарактеризуйте элементы физической культуры личности.}

Физическая культура личности – часть общей культуры человека, отражающая степень освоения и использования им ценностей физической культуры. Включает в себя:

\begin{enumerate}
    \item Физическое развитие: Один из основных элементов физической культуры личности - это физическое развитие
    \item Физическое образование: Элементом физической культуры личности является физическое образование, которое включает получение знаний о физической активности, спорте
    \item Спортивная деятельность: Спортивная деятельность является важным элементом физической культуры личности. Включает в себя участие в спортивных соревнованиях, тренировки в спортивных командах или индивидуальные занятия определенным видом спорта.
    \item Рекреационная активность: Рекреационная активность, такая как прогулки, бег, велосипедные прогулки, плавание и другие формы физической активности, также является важным элементом физической культуры личности
    \item Здоровый образ жизни: Здоровый образ жизни, включающий правильное питание
\end{enumerate}

\subsection{Охарактеризуйте задачи системы физической культуры.}

\begin{enumerate}
    \item Здоровье и оздоровление: Одной из главных задач системы физической культуры является сохранение и укрепление здоровья людей
    \item Физическое развитие и формирование физических качеств: Система физической культуры ставит перед собой задачу развития физических качеств у людей. Это включает развитие силы, выносливости, гибкости
    \item Формирование полноценной личности: Система физической культуры имеет задачу способствовать формированию полноценной личности. Это включает развитие не только физических, но и психологических, социальных и моральных аспектов личности
    \item Спортивные достижения и развитие спорта: Одной из задач системы физической культуры является развитие спорта и спортивных достижений. Это включает поддержку и развитие спортивных организаций, тренировку и подготовку спортсменов
    \item Развитие физической культуры в обществе: Система физической культуры также ставит перед собой задачу развития физической культуры в обществе. Это включает популяризацию и пропаганду здорового образа жизни
\end{enumerate}

\subsection{На каких основах возникает современная система физической культуре в обществе? Дайте краткую характеристику каждой.}

\begin{enumerate}
    \item Здоровье и благополучие: Одной из основных основ современной системы физической культуры является стремление к здоровью и благополучию
    \item Научные исследования и медицинские рекомендации: Современная система физической культуры опирается на научные исследования в области физической активности и здоровья
    \item Развитие спорта и спортивной инфраструктуры: Современная система физической культуры базируется на развитии спорта и спортивной инфраструктуры.
    \item Образование и освещение: Современная система физической культуры опирается на образование и освещение в области физической активности. В школах и учебных заведениях проводятся уроки физической культуры, где дети и молодежь получают знания и навыки для поддержания активного образа жизни
    \item Государственная поддержка и регулирование: Современная система физической культуры также базируется на государственной поддержке и регулировании
\end{enumerate}

\subsection{Назовите документы, определяющие программно-нормативные основы системы физической культуры.}

\begin{enumerate}
    \item Конституция Российской Федерации: В Конституции РФ закреплено право граждан на физическую культуру и спорт, а также государственная поддержка физической культуры и спорта.
    \item Федеральный закон "О физической культуре и спорте": Этот закон определяет правовые основы организации и развития физической культуры и спорта в Российской Федерации, устанавливает принципы государственной политики в данной области.
    \item Федеральные государственные образовательные стандарты: Образовательные стандарты определяют требования к содержанию и организации образовательного процесса, включая учебный предмет "Физическая культура".
    \item Нормативные акты Министерства спорта и Министерства образования: Это включает приказы и инструкции, которые устанавливают конкретные требования, методики, нормы и стандарты в области физической культуры и спорта, направленные на обеспечение качественной организации и проведения занятий, тренировок, соревнований и других мероприятий.
\end{enumerate}

\subsection{Каково основное содержание Государственного стандарта по физическому воспитанию в образовательной школе?}

\begin{enumerate}
    \item Цели и задачи: Стандарт определяет цели и задачи физического воспитания в школе, которые включают развитие физических качеств, формирование физической культуры личности, поддержание и укрепление здоровья, развитие двигательных навыков и умений, развитие моторики, координации и гибкости, развитие познавательных и эмоциональных сфер.
    \item Содержание образования: Стандарт определяет содержание физического воспитания, включая программы, темы и содержательные модули, которые должны быть освоены учащимися в разных возрастных группах. Оно охватывает такие области, как лёгкая атлетика, гимнастика, плавание, игровые виды спорта, физические упражнения, здоровьесберегающие технологии и другие.
    \item Организация образовательного процесса: Стандарт определяет организацию физического воспитания, включая распределение учебного времени, методики преподавания, использование специального оборудования и инвентаря, безопасность занятий, оценку и контроль уровня физической подготовки учащихся.
    \item Взаимодействие с другими предметами: Стандарт указывает на взаимодействие физического воспитания с другими предметами в учебном плане, такими как биология, физика, психология, социология и другие, для обеспечения комплексного подхода к физическому развитию и формированию личности.
    \item Учебно-методическое обеспечение: Стандарт предусматривает учебно-методическое обеспечение физического воспитания, включая разработку учебных программ, методических пособий, требований к квалификации педагогических работников, организацию учебных мероприятий, тренировок и соревнований.
\end{enumerate}

\subsection{Перечислите требования, которым должны отвечать разрабатываемые программы по физическому воспитанию.}

\begin{enumerate}
    \item Соответствие целям и задачам физического воспитания: Программы должны быть выстроены в соответствии с целями и задачами физического воспитания, которые включают развитие физических качеств, формирование физической культуры личности, поддержание и укрепление здоровья, развитие двигательных навыков и умений.
    \item Учет возрастных особенностей учащихся: Программы должны учитывать возрастные особенности учащихся, так как физическое развитие и способности детей различаются в зависимости от возрастных групп. Программы должны предусматривать разнообразные и адаптированные задания и упражнения для разных возрастов.
    \item Систематичность и постепенность: Программы должны быть построены с учетом принципа систематичности и постепенного увеличения нагрузки. Они должны предусматривать поэтапное развитие физических способностей и навыков учащихся, обеспечивая прогрессивное развитие и достижение поставленных целей.
    \item Учет индивидуальных особенностей и потребностей: Программы должны учитывать индивидуальные особенности и потребности учащихся. Это включает учет физического состояния, уровня подготовленности, интересов и мотивации каждого ученика, чтобы обеспечить их оптимальное развитие и максимальное вовлечение в физическую активность.
    \item Разнообразие видов физической активности: Программы должны предусматривать разнообразие видов физической активности, включая игровые виды спорта, гимнастику, лёгкую атлетику, плавание, танцы и другие формы двигательной деятельности.
    \item Безопасность и здоровье: Программы должны предусматривать меры безопасности при проведении занятий физической культурой.
\end{enumerate}

\subsection{Какие вы знаете программы по физическому воспитанию для общеобразовательных школ?}

\begin{enumerate}
    \item Программа "Физическая культура" (Россия): Одна из основных программ, разработанная на основе Федерального государственного образовательного стандарта. Она определяет содержание и организацию физического воспитания в школе, включая различные виды спорта, гимнастику, лёгкую атлетику и другие формы физической активности.
    \item Программа "Физическое воспитание и спорт" (США): В Соединенных Штатах существует несколько программ, включая "Программу физического воспитания для школ" (Physical Education for Progress Act) и "Программу школьного спорта" (School Sports Program), которые определяют стандарты и содержание физического воспитания в школах.
    \item Программа "Физическая культура и спорт" (Великобритания): В Великобритании существует национальная программа по физическому воспитанию и спорту, которая включает в себя разнообразные виды физической активности, включая игры, спортивные мероприятия и физическую подготовку.
    \item Программа "Физическое воспитание и здоровье" (Австралия): В Австралии физическое воспитание и здоровье являются обязательными предметами в школьной программе. Программа включает физическую активность, спорт, здоровый образ жизни и обучение навыкам поддержания физического и психического благополучия.
    \item Программа "Физическая культура и спорт" (Канада): В Канаде существует национальная программа по физическому воспитанию и спорту, которая определяет стандарты и цели физического воспитания в школах. Программа включает различные виды спорта, игры, физическую подготовку и активные формы отдыха.
\end{enumerate}

\subsection{Каково основное содержание Единой спортивной классификации?}

\begin{enumerate}
    \item Квалификационные разряды: ЕСК определяет различные квалификационные разряды для спортсменов, которые характеризуют их уровень подготовки и достижений в спортивной дисциплине.
    \item Спортивные звания: ЕСК может определять спортивные звания, которые присваиваются спортсменам за особые достижения и выдающиеся результаты в своей спортивной дисциплине.
    \item Нормативы и требования: ЕСК устанавливает нормативы и требования, которым спортсмены должны соответствовать для получения определенного разряда или звания.
    \item Процедуры аттестации: ЕСК определяет процедуры и условия аттестации спортсменов на получение квалификационных разрядов и спортивных званий. Обычно включаются такие этапы, как сдача нормативов, участие в соревнованиях определенного уровня, проверка медицинской готовности и т.д.
    \item Учет результатов и регистрация: ЕСК предусматривает учет и регистрацию достижений спортсменов, включая результаты соревнований, полученные разряды и звания.
\end{enumerate}

\subsection{Дайте определения понятиям ``движение'', ``действие'', ``физическое упражнение''.}

\begin{enumerate}
    \item Движение: Движение — это изменение положения тела или его частей в пространстве с течением времени. Оно характеризуется перемещением объектов, изменением их положения, формы или ориентации. Движение может быть как естественным (например, движение планет, течение рек), так и вызванным действием внешних сил или усилий человека.
    \item Действие: Действие — это акт или процесс совершения какого-либо действия или воздействия на объекты или окружающую среду. Действие может быть физическим или психическим, целенаправленным или случайным. Оно может иметь различные цели или последствия и осуществляться с помощью физических движений, слов, мыслей или других средств воздействия.
    \item Физическое упражнение: Физическое упражнение — это специально организованная деятельность, которая выполняется с целью улучшения физической формы, развития физических качеств или поддержания здоровья. Оно включает в себя систематическое выполнение физических движений и упражнений, таких как физические нагрузки, тренировки, спортивные занятия и т.д. Физические упражнения могут быть направлены на развитие силы, выносливости, гибкости, координации и других физических качеств, а также на достижение конкретных спортивных целей.
\end{enumerate}

\subsection{Почему физические упражнения являются основным средством физической культуры?}

\begin{enumerate}
    \item Развитие физических качеств: Физические упражнения способствуют развитию физических качеств, таких как сила, выносливость, гибкость, координация, баланс и скорость. Они направлены на улучшение функциональности организма и формирование здорового и сильного тела.
    \item Укрепление здоровья: Регулярные физические упражнения способствуют укреплению здоровья. Они улучшают работу сердечно-сосудистой системы, повышают иммунитет, улучшают обмен веществ, способствуют снижению риска развития многих заболеваний, включая сердечно-сосудистые заболевания, ожирение и диабет.
    \item Формирование навыков и умений: Физические упражнения позволяют формировать навыки и умения, связанные с определенными видами физической активности. Например, спортивные упражнения помогают развивать спортивные навыки, тактику и стратегию, а также улучшают соревновательные навыки.
    \item Психологический эффект: Физические упражнения оказывают положительное влияние на психическое состояние человека. Они способствуют выработке эндорфинов - гормонов счастья, которые улучшают настроение и снижают уровень стресса и тревожности. Регулярные тренировки также способствуют повышению самооценки и самодисциплине.
    \item Социальная составляющая: Физические упражнения могут быть проведены в групповой форме, что способствует социализации и взаимодействию с другими людьми. Это создает возможность для общения, установления социальных связей и развития командного духа.
\end{enumerate}

\subsection{Перечислите факторы, определяющие эффективность воздействия физических упражнений.}

\begin{enumerate}
    \item Индивидуальные особенности: Каждый человек уникален и имеет свои индивидуальные особенности, такие как возраст, пол, физическая подготовленность, здоровье и особенности организма. Эти факторы могут влиять на способность организма к адаптации к физическим упражнениям и их эффективность.
    \item Тип физической активности: Различные виды физической активности могут иметь разные цели и эффекты. Например, аэробные упражнения направлены на улучшение кардиореспираторной выносливости, силовые тренировки на развитие силы и мышц, гибкостные упражнения на увеличение гибкости и т.д. Эффективность воздействия будет зависеть от выбранного типа физической активности и соответствия ее целям и потребностям организма.
    \item Интенсивность и объем тренировок: Интенсивность физических упражнений отражает уровень нагрузки, которую испытывает организм. Оптимальная интенсивность будет зависеть от физической подготовленности и целей тренировок. Объем тренировок отражает количество выполненных упражнений или продолжительность тренировок. Эффективность воздействия физических упражнений может быть связана с правильной балансировкой интенсивности и объема тренировок.
    \item Регулярность и постоянство: Регулярность физических упражнений играет важную роль в достижении результатов. Постоянные тренировки способствуют прогрессу и улучшению физической формы. Непрерывность в практике физической активности позволяет организму адаптироваться и получить долгосрочные результаты.
    \item Правильная техника выполнения: Правильная техника выполнения упражнений играет ключевую роль в предотвращении травм и максимизации выгод от тренировок.
\end{enumerate}

\subsection{Чем характеризуется содержание и форма физических упражнений?}

\textbf{Содержание физических упражнений:}
\begin{enumerate}
    \item Цели и задачи: Физические упражнения могут быть направлены на достижение различных целей, таких как улучшение физической формы, развитие определенных физических качеств (силы, выносливости, гибкости), улучшение координации движений и т.д. Содержание упражнений определяется целями, которые хотят достичь практикующие.
    \item Типы упражнений: Существует множество типов физических упражнений, включая аэробные упражнения (бег, плавание, езда на велосипеде), силовые тренировки (поднятие гирь, отжимания), гибкостные упражнения (растяжка, йога) и упражнения для развития координации и баланса. Содержание упражнений может варьироваться в зависимости от выбранного типа физической активности.
\end{enumerate}

\textbf{Форма физических упражнений:}
\begin{enumerate}
    \item Упражнения с использованием оборудования: Это упражнения, выполняемые с использованием специального оборудования или тренажеров, например, тренажерный зал, гантели, гимнастические приспособления и т.д. Форма таких упражнений определяется конкретным оборудованием, которое используется для выполнения упражнений.
    \item Упражнения без оборудования: Это упражнения, которые могут быть выполнены без использования специального оборудования. К ним относятся, например, бег, прыжки, отжимания, приседания, растяжка и другие упражнения, которые можно выполнять собственным весом тела или с минимальным использованием простых предметов, таких как гимнастический мат или скакалка.
    \item Упражнения в группе или индивидуальные упражнения: Физические упражнения могут выполняться в групповой форме, где участники занимаются одновременно под руководством тренера или инструктора.
\end{enumerate}

\subsection{Приведите примеры классификации физических упражнений по 5-6 классификационным признакам.}

\begin{enumerate}
    \item По типу физической активности:
          \begin{itemize}
              \item Аэробные упражнения (бег, плавание, езда на велосипеде).
              \item Силовые тренировки (поднятие гирь, отжимания, приседания).
              \item Гибкостные упражнения (растяжка, йога).
              \item Упражнения для развития координации и баланса (акробатика, танцы).
          \end{itemize}

    \item По функциональной направленности:
          \begin{itemize}
              \item Упражнения для развития силы.
              \item Упражнения для развития выносливости.
              \item Упражнения для развития гибкости.
              \item Упражнения для развития координации и баланса.
          \end{itemize}

    \item По месту проведения:
          \begin{itemize}
              \item Упражнения в тренажерном зале.
              \item Упражнения на открытом воздухе.
              \item Упражнения в бассейне или на воде.
              \item Упражнения в спортивном зале.
          \end{itemize}

    \item По интенсивности:
          \begin{itemize}
              \item Низкоинтенсивные упражнения (прогулка, легкая йога).
              \item Среднеинтенсивные упражнения (бег темпом, плавание).
              \item Высокоинтенсивные упражнения (интервальные тренировки, быстрый бег).
          \end{itemize}

    \item По группам мышц, задействованных в упражнении:
          \begin{itemize}
              \item Упражнения для верхней части тела (отжимания, подтягивания).
              \item Упражнения для нижней части тела (приседания, выпады).
              \item Упражнения для ядра и корпуса (планка, пресс).
          \end{itemize}

    \item По спортивным дисциплинам:
          \begin{itemize}
              \item Упражнения в баскетболе (дриблинг, броски).
              \item Упражнения в гимнастике (сальто, вис на перекладине).
              \item Упражнения в плавании (брасс, кроль).
          \end{itemize}
\end{enumerate}

\subsection{Что понимается под техникой физических упражнений?}

Техника физических упражнений относится к правильному и эффективному выполнению конкретного упражнения или движения. Она включает в себя правильную позицию тела, координацию движений, контроль над мышцами и дыханием. Техника играет важную роль в достижении максимальной пользы от физических упражнений и предотвращении возможных травм.

Важные аспекты техники физических упражнений включают следующее:
\begin{enumerate}
    \item Правильная позиция тела: Это включает в себя правильную осанку, правильное выравнивание позвоночника и других частей тела. Например, во время выполнения приседаний важно сохранять прямую спину и сохранять правильное положение коленей.
    \item Координация движений: Это связано с согласованным и гармоничным выполнением движений различных частей тела. Например, во время выполнения упражнений на тренажере с использованием рук и ног, необходимо обеспечить согласованное движение и контроль над всеми частями тела.
    \item Контроль мышц: Это относится к контролю над активируемыми мышцами во время выполнения упражнений. Например, во время поднятия гирь важно активировать и контролировать целевые мышцы, чтобы максимально задействовать их и избежать перенапряжения других групп мышц.
    \item Дыхание: Правильное дыхание является важной частью техники физических упражнений. Синхронизация дыхания с движениями помогает улучшить эффективность и контроль над упражнением. Например, во время подъема веса рекомендуется выдохнуть на усилии и вдохнуть во время расслабления.
\end{enumerate}

Правильная техника физических упражнений позволяет достичь наилучших результатов, минимизировать травмы и максимально использовать потенциал физической активности.

\subsection{Что понимается под пространственными, временными, пространственно-временными, силовыми и ритмическими характеристиками техники физических упражнений?}

\begin{enumerate}
    \item Пространственные характеристики техники физических упражнений относятся к описанию и анализу движений в пространстве. Это включает положение и движение различных частей тела во время выполнения упражнений. Например, положение рук, ног, тела и их движение в пространстве во время выполнения определенного движения.
    \item Временные характеристики техники физических упражнений относятся к описанию и анализу временной структуры движений. Это включает скорость и темп выполнения упражнений, продолжительность отдельных фаз и периодов движений. Например, время, затраченное на определенную фазу движения или интервалы времени между повторениями упражнений.
    \item Пространственно-временные характеристики техники физических упражнений связаны с описанием и анализом взаимодействия пространственных и временных параметров движений. Это включает координацию движений в пространстве и согласованное управление временными параметрами. Например, синхронизация движений рук и ног или поддержание определенного ритма и темпа движений.
    \item Силовые характеристики техники физических упражнений связаны с силовым воздействием, создаваемым различными мышечными группами во время выполнения упражнений. Это включает описание силы, направления и интенсивности мышечной активности. Например, сила при поднятии гирь или при выполнении отжиманий.
    \item Ритмические характеристики техники физических упражнений относятся к описанию и анализу ритма и темпа движений. Это включает определение и поддержание определенного ритма, регулярность и гармоничность движений. Например, ритмичное выполнение упражнений в соответствии с музыкальным сопровождением или синхронизация движений в групповых упражнениях, таких как аэробика или танцы.
\end{enumerate}
