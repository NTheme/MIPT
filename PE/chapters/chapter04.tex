\section{Семестр}

\subsection{Дайте определение понятиям «обучение», «обучаемость», «знания».}



\subsection{Какие виды знаний Вам известны?}



\subsection{Дайте характеристику двигательных умений.}



\subsection{Опишите фазы формирования двигательного навыка.}



\subsection{Что Вы понимаете под ориентировочной основой действия?}



\subsection{Что Вам известно о взаимодействии (переносе) навыков?}



\subsection{Дайте определение двигательной ошибке.}



\subsection{Перечислите основные причины ошибок и пути их устранения.}



\subsection{Назовите этапы обучения двигательным действиям, цель и задачи на каждом этапе.}



\subsection{Дайте определения понятий «физические способности», «физические качества».}



\subsection{Раскройте понятия «физические способности», «физические качества».}



\subsection{Перечислите врожденные задатки, которые лежат в основе развития физических способностей.}



\subsection{На какие классы разделил В.И. Лях все физические способности?}



\subsection{Перечислите факторы, от которых зависит развитие кондиционных способностей.}



\subsection{Какими механизмами обусловлен комплекс координационных способностей?}



\subsection{Перечислите основные закономерности развития физических способностей.}



\subsection{Какой фактор является ведущим в развитии физических способностей?}



\subsection{Охарактеризуйте основные режимы двигательной активности, зависящие от того, в какой фазе отдыха повторяется каждое последующее упражнение.}



\subsection{Дайте определение сенситивных периодов.}



\subsection{В чем проявляется сущность этапности развития физических способностей?}



\subsection{Проиллюстрируйте на примере закономерность переноса физических способностей.}



\subsection{В чем особенности единства и взаимосвязи между двигательными умениями и физическими способностями?}



\subsection{Перечислите основные принципы развития физических способностей.}
