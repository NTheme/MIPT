\section{Семестр}

\subsection{Раскройте принцип регулярности педагогических воздействий.}

Процессы обучения движениям и развития физических способностей подчиняются разным закономерностям.
Принципы развития физических способностей выражают закономерности взаимосвязи состояния человека и физической нагрузки
в зависимости от ее организации во времени.

Принцип регулярности педагогических воздействий предполагает необходимость постоянных занятий физическими упражнениями
для развития физических способностей человека.
В результате многократного выполнения двигательных действий в отдельном занятии и повторяемости самих занятий
в организме человека происходят функциональные сдвиги, которые характеризуют соответствующий эффект.
При практической реализации принципа регулярности важно обеспечить постоянство, непрерывность
адаптационных перестроек функционального и структурного характера, составляющих биологическую основу развития
физических способностей.


\subsection{Какие эффекты происходят в организме человека в результате выполнения физических упражнений?}

\begin{enumerate}
    \item \textbf{Срочный (ближайший) эффект} --- это изменения, наступившие в организме после выполнения каждого упражнения или к моменту завершения занятия. Этот эффект не исчезает сразу, а сохраняется некоторое время.
    \item \textbf{Трансформированный (отставленный) эффект} --- это изменения в состоянии организма, наблюдаемые после окончания предыдущего занятия до начала очередного занятия. Если между занятиями следует слишком большой перерыв, то данный эффект может исчезнуть. Поэтому для прогрессивного изменения показателей физических способностей педагог должен строить педагогический процесс так, чтобы «следы» от каждого предыдущего занятия наслаивались на эффект последующего. Благодаря такому сложению возникает кумулятивный (накопительный) эффект.
    \item \textbf{Кумулятивный (накопительный) эффект} --- это накопление. Суммирование эффектов от тренировочных занятий.
\end{enumerate}


\subsection{В чем заключается принцип прогрессирования и адаптивно-адекватной предельности в наращивании эффекта педагогических воздействий?}

Данный принцип базируется на закономерностях этапности и неравномерности развития физических способностей
(другими словами, постепенности и предельности).

Постепенность означает плавное увеличение нагрузки как в отдельном, так и в целом ряде занятий,
облегчает приспособление организма к ним, содействует углублению и закреплению вызванных ими адаптационных перестроек и
тем самым способствует созданию предпосылок перехода на новый, повышенный уровень нагрузок.

Постепенность в повышении нагрузок предполагает применение предельных (максимальных) нагрузок.
Предельной нагрузкой считают такую, которая в полной мере мобилизует функциональные резервы организма человека и
не приводит к его перенапряжению и перетренировке. Понятие «предельная» нагрузка имеет относительный характер:
то, что является предельной нагрузкой при одном уровне подготовленности, перестает быть таковой при другом.


\subsection{Раскройте значение принципа рационального сочетания и распределения во времени педагогических воздействий
    различного характера.}

Этот принцип требует соблюдения разумно обоснованного, целесообразного способа взаимосвязи и порядка следования различных
по величине нагрузок как внутри отдельного занятия, так и в рамках серии занятий. Для его реализации важное значение имеет
учет закономерностей «переноса» физических способностей и закономерностей, лежащих в основе чередования нагрузки и отдыха.

В процессе развития физических способностей могут использоваться нагрузки избирательного (однонаправленного) и
комплексного характера. Первые предусматривают преимущественное развитие отдельных способностей (скоростных, силовых и др.),
а вторые обеспечивают последовательное или параллельное (одновременное) совершенствование разных способностей, скажем,
скоростных возможностей и выносливости.

Доказано, что использование однонаправленных нагрузок при развитии одной какой-либо способности оказывает на организм человека
более глубокое, но локальное воздействие.

Нагрузка комплексной направленности оказывает более широкое, но менее глубокое воздействие на организм.
При использовании нагрузок комплексной направленности с последовательным развитием различных способностей в первую очередь
необходимо:
\begin{enumerate}
    \item Определить рациональную последовательность, т.е. порядок и очередность введения в занятие нагрузок,
          соответствующих развитию разных способностей.
    \item Выбрать рациональное соотношение объема и интенсивности нагрузок.
\end{enumerate}

Доказано, что нагрузки скоростного характера создают благоприятный физиологический фон для нагрузок, требующих проявления выносливости. Последние же оставляют за собой фон, который в течение ряда часов может неблагоприятно сказываться на выполнении скоростных упражнений. Установлено также, что скоростные нагрузки хорошо сочетаются с воздействиями силового характера. Поэтому при выборе оптимальной последовательности нагрузки в комплексных занятиях целесообразно придерживаться следующего порядка применения по их преимущественной направленности:

\begin{equation*}
    \begin{array}{lcr}
        \text{СИЛОВЫЕ}    & \text{СКОРОСТНЫЕ} & \text{НА ВЫНОСЛИВОСТЬ} \\
                          & \text{или}        &                        \\
        \text{СКОРОСТНЫЕ} & \text{СИЛОВЫЕ}    & \text{НА ВЫНОСЛИВОСТЬ}
    \end{array}
\end{equation*}

Вопрос о соотношении объема и интенсивности нагрузок в каждом конкретном случае должен решаться с учетом характера,
направленности и последовательности применения, функционального состояния, индивидуальных особенностей занимающихся и т.п.


\subsection{Дайте характеристику принципа возрастной адекватности педагогических воздействий.}

Обязывает педагога осуществлять формирование способностей в соответствии с естественно сменяющимися периодами онтогенеза
занимающегося. Зная критические (сенситивные) периоды в формировании той или иной способности, возможно направленно и
эффективно влиять на уровень их развития.


\subsection{Дайте определение понятий ``мышечная сила'', ``абсолютная сила'', ``относительная сила'', ``взрывная сила''.}

Выполнение любого движения или сохранения какой-либо позы тела человека обусловлено работой мышц. Величину развиваемого при
этом усилия принято называть силой мышц.

\textbf{Мышечная сила} --- это способность человека преодолевать внешнее сопротивление или противодействовать ему за счет мышечных
напряжений.

Различают следующие виды силовых способностей: собственносиловые, и их соединение с другими физическими способностями:

\begin{enumerate}
    \item \textbf{Собственно-силовые способности} проявляются в условиях статического режима и медленных движений (например, при удержании
          предельных отягощений с максимальным напряжением мышц или при перемещении предметов большой массы).
          Для оценки степени развития собственно-силовых способностей различают абсолютную и относительную силу действия человека.

          \textbf{Абсолютная сила} определяется максимальными показателями мышечных напряжений без учета массы тела человека.

          \textbf{Относительная сила} --- отношением величины абсолютной силы к собственной массе тела, т.е. величиной силы, приходящейся на 1 кг
          собственного веса тела.

          У людей, имеющих примерно одинаковый уровень тренированности, повышение массы тела ведет к увеличению абсолютной силы, но
          при этом величина относительной силы снижается. Выделение абсолютной и относительной силы действия имеет большое практическое
          значение. Так, достижения спортсменов самых тяжелых весовых категорий в тяжелой атлетике, спортивных единоборствах,
          а также при метаниях спортивных снарядов определяются прежде всего уровнем развития абсолютной силы.
          В видах деятельности с большим количеством перемещений тела в пространстве (например, в гимнастике) или имеющих ограничения
          массы тела (например, весовые категории в борьбе) успешность во многом будет зависеть от развития относительной силы.

          Результаты исследований позволяют утверждать, что уровень абсолютной силы человека в большей степени обусловлен факторами
          среды (тренировка, регулярные занятия и др.). В то же время показатели относительной силы в большей мере испытывают на себе влияние генотипа.

    \item \textbf{Скоростно-силовые способности} проявляются в двигательных действиях, в которых наряду со значительной силой мышц требуется и
          значительная быстрота движений (прыжки в длину и высоту с места и разбега, метания снарядов и т.п.). При этом чем выше внешнее
          отягощение, (например, при толкании ядра или выполнение рывка гири достаточно большого веса), тем большую роль играет силовой компонент,
          а при меньшем отягощении (например, при метании малого мяча) возрастает значимость скоростного компонента.

          Важной разновидностью скоростно-силовых способностей является \textbf{взрывная сила} - способность проявлять большие величины
          силы в наименьшее время (например, при старте в спринтерском беге, в прыжках, метаниях и т.д.).
          Уровень развития взрывной силы можно оценить с помощью скоростно-силового индекса, который вычисляется по формуле:

          \begin{equation}
              J = \frac{F_{\text{max}}}{t_{\text{max}}}
          \end{equation}
          где:
          \begin{itemize}
              \item $J$ – скоростно-силовой индекс;
              \item $F_{\text{max}}$ – максимальное значение силы, показанное в данном
                    движении;
              \item $t_{\text{max}}$ – время достижения максимальной силы.
          \end{itemize}

    \item \textbf{Силовая ловкость} --- способность точно дифференцировать мышечные усилия различной величины в условиях
          непредвиденных ситуаций и смешанных режимов работы мышц. Силовая ловкость проявляется там, где есть сменный характер
          режима работы мышц, меняющиеся и непредвиденные ситуации деятельности (регби, борьба, хоккей).
\end{enumerate}


\subsection{Назовите режимы работы мышц.}

Одним из наиболее существенных моментов, определяющих мышечную силу, является \textbf{режим работы мышц}. В процессе выполнения
двигательных действий мышцы могут проявлять силу:

\begin{enumerate}
    \item при уменьшении своей длины (\textbf{преодолевающий}, т.е. \textbf{миометрический режим}, например,
          жим штанги лежа на горизонтальной скамейке)
    \item при ее удлинении (\textbf{уступающий}, т.е. \textbf{полиометрический режим}, например, приседание со штангой на плечах)
    \item без изменения своей длины (\textbf{статический}, т.е. \textbf{изометрический режим}, например, удержание разведенных рук
          с гантелями в наклоне вперед)
    \item при изменении и длины и напряжения мышц (\textbf{смешанный}, т.е. \textbf{ауксотонический режим}, например, подъем силой в упор
          на кольцах, опускание в упор руки в стороны («крест») и удержание в «кресте»)
\end{enumerate}

Первые два режима характерны для динамической, третий – для статической, четвертый – для статодинамической работы мышц.

В любом режиме работы мышц сила может быть проявлена медленно и быстро. Это характер их работы.


\subsection{Перечислите механизмы, обеспечивающие проявление силовых способностей.}

К физиологическим механизмам развития силы можно отнести следующие факторы:
\begin{enumerate}
    \item Внутримышечные:
    \begin{itemize}
        \item Величина физиологического поперечника. Чем поперечник толще, тем большее усилие могут развить мышцы. При рабочей гипертрофии мышц в мышечных волокнах увеличивается количество и размеры миофибрилл (сократительные волокна) и повышается концентрация саркоплазматических белков.
        \item Состав (композиция) мышечных волокон. Различают «медленные» и «быстрые» мышечные волокна. Первые развивают меньшую мышечную силу напряжения, причем со скоростью в три раза меньшей, чем «быстрые» волокна. Второй тип волокон осуществляет быстрые и мощные сокращения. Силовая тренировка с большим весом отягощения и небольшим числом повторений мобилизует значительное количество «быстрых» мышечных волокон, в то время как занятия с небольшим весом и большим количеством повторений активизирует как «быстрые», так и «медленные» волокна. В различных мышцах тела соотношение волокон неодинаково, и генетически обусловлен.
        \item На силу мышечного сокращения влияют эластичные свойства, вязкость, анатомическое строение, структура мышечных волокон и их химический состав.
    \end{itemize}
    \item Особенности нервной регуляции:
    \begin{itemize}
        \item Частотой нервных импульсов, поступающих в скелетные мышцы от мотонейронов спинного мозга и обеспечивающих переход от слабых одиночных сокращений волокон к более сильным и мощным.
        \item Активизацией многих двигательных единиц (ДЕ). При увеличении числа вовлеченных ДЕ повышается сила сокращения мышцы.
        \item Синхронизацией активности ДЕ. Одновременное сокращение большего числа ДЕ резко увеличивает силу мышц.
        \item Межмышечной координацией. Сила мышцы зависит от деятельности других мышечных групп: сила мышцы растет при одновременном расслаблении ее антагониста, она уменьшается при одновременном сокращении других мышц и увеличивается при фиксации туловища или отдельных суставов мышцами-антагонистами. Например, при подъеме штанги возникает явление натуживания (выдох при закрытой голосовой щели), приводящее к фиксации мышцами туловища спортсмена и создающее прочную основу для преодоления поднимаемого веса.
    \end{itemize}
    \item Психофизиологические механизмы увеличении мышечной силы связаны с изменениями функционального состояния (бодрости, сонливости, утомления), а также влияниями мотиваций и эмоций.
\end{enumerate}

Важную роль в развитии силы играют мужские половые гормоны (андрогены), которые обеспечивают рост синтеза сократительных белков в скелетных мышцах. Их у мужчин в 10 раз больше, чем у женщин. Этим объясняется больший тренировочный эффект развития силы у спортсменов по сравнению со спортсменками, даже при абсолютно одинаковых тренировочных нагрузках.

Максимальная сила, которую может проявить человек, зависит и от механических особенностей движения. К ним относятся: исходное положение (или поза), длина плеча рычага и изменение угла тяги мышц, состояние мышцы перед сокращением (предварительно растянутая мышца сокращается сильно и быстро) и т.д.

Сила увеличивается под влиянием предварительной разминки и соответствующего повышения возбудимости ЦНС до оптимального уровня. И наоборот, чрезмерное возбуждение и утомление могут уменьшить максимальную силу мышц.

Силовые возможности зависят от возраста и пола занимающихся. Самыми благоприятными периодами развития силы у мальчиков и юношей считается возраст от 13—14 до 17—18 лет, а у девочек и девушек — от 11—12 до 15—16 лет, чему в немалой степени соответствует доля мышечной массы к общей массе тела (к 10—11 годам она составляет примерно 23\%, к 14—15 годам — 33\%, а к 17—18 годам — 45\%).

Наиболее значительные темпы возрастания относительной силы различных мышечных групп наблюдаются в младшем школьном возрасте, особенно у детей от 9 до 11 лет. Пик проявления силовых способностей приходится на возраст 25-30 лет.

В проявлении силы наблюдается известная суточная периодика: ее показатели достигают максимальных величин между 15-16 часами. Отмечено, что в январе и феврале мышечная сила нарастает медленнее, чем в сентябре и октябре, что по-видимому, объясняется большим потреблением осенью витаминов и действием ультрафиолетовых лучей. Наилучшие условия для деятельности мышц – при температуре +20°C.


\subsection{Назовите группы упражнений, которые используются для развития силовых способностей.}

При развитии силовых способностей пользуются упражнениями с повышенным сопротивлением – силовыми упражнениями. В зависимости от природы сопротивления они подразделяются на 3 группы:
\begin{enumerate}
    \item Упражнения с внешним сопротивлением
    \begin{itemize}
        \item упражнения с тяжестями (штангой, гантелями, гирями), в том числе и на тренажерах;
        \item упражнения с сопротивлением других предметов (резиновых амортизаторов, жгутов, блочных устройств и др.);
        \item упражнения в преодолении сопротивления внешней среды (бег по песку, снегу, против ветра и т.п.).
    \end{itemize}
    \item Упражнения с преодолением собственного тела.
    \begin{itemize}
        \item гимнастические силовые упражнения (сгибание и разгибание рук в упорах, лазание по канату, поднимание ног к перекладине);
        \item легкоатлетические прыжковые упражнения (прыжки на одной или двух ногах, «в глубину»);
        \item упражнения в преодолении препятствий.
    \end{itemize}
    \item Изометрические упражнения, как никакие другие, способствуют одновременному напряжению максимально возможного количества двигательных единиц работающих мышц. Они подразделяются на:
    \begin{itemize}
        \item удержание в пассивном напряжении мышц (удержание груза на предплечьях рук, плечах, спине и т.п.);
        \item упражнения в активном напряжении мышц в течение определенного времени в определенной позе (выпрямление полусогнутых ног, попытка оторвать от пола штангу чрезмерного веса и т.п.).
    \end{itemize}
    Выполняемые обычно при задержке дыхания, они приучают организм к работе в очень трудных бескислородных условиях. Занятия с использованием изометрических упражнений требуют мало времени, оборудование для их проведения весьма простое и с помощью данных упражнений можно воздействовать на любые мышечные группы.
\end{enumerate}


\subsection{Дайте краткую характеристику методов развития силовых способностей.}

Направленное развитие силовых способностей происходит лишь тогда, когда осуществляются максимальные мышечные напряжения. Поэтому основная проблема в методике силовой подготовки состоит в том, чтобы обеспечить в процессе выполнения упражнений достаточно высокую степень мышечных напряжений. В методическом плане существуют различные способы создания максимальных напряжений: поднятие предельного веса небольшое количество раз; поднятие непредельного веса максимальное число раз; поднятие непредельного отягощения с максимальной скоростью; преодоление внешних сопротивлений при постоянной длине мышц; изменение ее тонуса при постоянной скорости движения; стимулирование сокращения мышц в суставе за счет энергии падающего груза или веса собственного тела и др. В соответствии с указанными способами стимулирования мышечных напряжений выделяют следующие методы развития силовых способностей:
\begin{enumerate}
    \item Метод максимальных усилий. Он основан на использовании упражнений с субмаксимальными, максимальными и сверхмаксимальными отягощениями. Каждое упражнение выполняется в несколько подходов. Количество повторений упражнений в одном подходе при преодолении предельных и сверхпредельных сопротивлений (когда вес отягощения равен 100\% и более) может составлять 1-2, максимум 3 раза. Число подходов 2-3, паузы отдыха между повторениями в подходе 3-4 минуты, а между подходами от 2 до 5 минут. При выполнении упражнений с околопредельными отягощениями (вес отягощения 90-95\% от максимального) число возможных повторений движений в одном подходе 5-6, количество подходов 2-5. интервалы отдыха между повторениями упражнений в каждом подходе – 4-6 мин и подходами 2-5 мин. Темп движений – произвольный, скорость – от малой до максимальной. В практике встречаются различные варианты этого метода, в основе которых лежат разные способы повышения отягощения в подходах. Данный метод обеспечивает повышение максимальной динамической силы без существенного увеличения мышечной массы. Рост силы при его использовании происходит за счет совершенствования внутри- и межмышечной координации и повышения мощности креатинфосфатного и гликолитического механизмов ресинтеза АТФ. Следует иметь ввиду, что предельные нагрузки затрудняют самоконтроль за техникой действий, увеличивают риск травматизма. Этот метод применяется 2-3 раза в неделю.
    \item Метод повторных непредельных усилий. Предусматривает многократное преодоление непредельного внешнего сопротивления до значительного утомления или до «отказа». В каждом подходе упражнение выполняется без пауз отдыха. В одном подходе может быть от 4 до 15-20 и более повторений упражнений. За одно занятие выполняется 2-6 серии. В серии 2-4 подхода. Отдых между подходами 2-8 мин, между сериями – 3-5 мин. Величина внешних сопротивлений обычно находится в пределах 40-80\% от максимальной. Скорость движений невысокая. Значительный объем мышечной работы с непредельными отягощениями активизирует обменно-трофические процессы в мышечной и других системах организма, вызывая необходимую гипертрофию мышц с увеличением их физиологического поперечника, стимулируя тем самым развитие максимальной силы. Необходимо отметить тот факт, что сила сохраняется дольше, если одновременно с ее развитием увеличивается и мышечная масса. Данный метод получил широкое распространение в практике, т.к. позволяет контролировать технику движений, избегать травм, уменьшать натуживание во время выполнения силовых упражнений, содействует гипертрофии мышц и является единственно возможным при подготовке начинающих.
    \item Метод изометрических усилий. Характеризуется выполнением кратковременных максимальных напряжений, без изменения длины мышц. Продолжительность изометрического напряжения обычно 5-10 с. Величина развиваемого усилия может быть 40-50\% от максимума и статические силовые комплексы должны состоять из 5-10 упражнений, направленных на развитие силы различных мышечных групп. Каждое упражнение выполняется 3-5 раз с интервалом отдыха 30-60 с. Изометрические упражнения целесообразно включать в занятия до 4 раз в неделю, отводя на них каждый раз по 10-15 мин. Комплекс упражнений применяется в неизменном виде примерно в течение 4-6 недель, затем он обновляется. Паузы отдыха заполняются выполнением упражнений на дыхание, расслабление и растяжение. При выполнении изометрических упражнений важное значение имеет выбор позы или величины суставных углов. Так, например, изометрические напряжения при 90° оказывают большое влияние на прирост динамической силы, чем при углах 120° и 150°. Недостаток изометрических упражнений состоит в том, что сила проявляется в большей мере при тех суставных углах, при которых выполнялись упражнения, а уровень силы удерживается меньшее время, чем после динамических упражнений.
    \item Метод изокинетических усилий. Специфика этого метода состоит в том, что при его использовании задается не величина внешнего сопротивления, а постоянная скорость движения. Это дает возможность работать мышцам с оптимальной нагрузкой на протяжении всего движения, чего нельзя добиться, применяя любые из общепринятых методов. Чаще всего упражнения выполняются на специальных тренажерах. Этот метод используется для развития различных типов силовых способностей – «медленной», «быстрой», «взрывной» силы. Он обеспечивает значительное увеличение силы за более короткий срок по сравнению с методами повторных и изометрических усилий. Силовые занятия, основанные на выполнении упражнений изокинетического характера, исключают возможность получения мышечно-суставных травм.
    \item Метод динамических усилий. Предусматривает выполнение упражнений с относительно небольшой величиной отягощений (до 30\% от максимума) и максимальной скоростью. Он применяется для развития скоростно-силовых способностей. Количество повторений упражнения в одном подходе составляет 15-20 раз. Упражнения выполняются в 3-6 серий, с отдыхом между ними 5-8 минут. Вес отягощения в каждом упражнении должен быть таким, чтобы он не оказывал существенных нарушений в технике движений и не приводил к замедлению скорости выполнения двигательного задания.
    \item Ударный метод основан на ударном стимулировании мышечных групп путем использования кинетической энергии падающего груза или веса собственного тела (прыжки в глубину с последующим выпрыгиванием вверх, в том числе и с отягощениями). Поглощение тренирующими мышцами энергии падающей массы способствует резкому переходу мышц к активному состоянию, быстрому развитию рабочего усилия, создает в мышце дополнительный потенциал напряжения, что обеспечивает значительную мощность и быстроту отталкивающего движения и быстрый переход от уступающей работы к преодолевающей. Этот метод применяется для развития «амортизационной» и «взрывной» силы различных мышечных групп.
    \item Метод круговой тренировки. Обеспечивает комплексное воздействие на различные мышечные группы. Упражнения проводятся по станциям и подбираются таким образом, чтобы каждая последующая серия включала в работу новую группу мышц. Число упражнений, воздействующих на разные группы мышц, продолжительность их выполнения на станциях зависят от задач, решаемых в тренировочном процессе, возраста, пола и подготовленности занимающихся. Комплекс упражнений с использованием непредельных отягощений повторяют 1-3 раза по кругу. Отдых между каждым повторением комплекса должен составлять не менее 2-3 мин, во время которого выполняются упражнения на расслабление.
    \item Игровой метод предусматривает воспитание силовых способностей преимущественно в игровой деятельности, где игровые ситуации вынуждают менять режимы напряжения различных мышечных групп и бороться с нарастающим утомлением организма. К таким играм относятся игры, требующие удержания внешних объектов (например, партнера в игре «Всадники»), игры с преодолением внешнего сопротивления (например, «Перетягивание каната»), игры с чередованием режимов напряжения различных мышечных групп (например, различные эстафеты с переноской грузов различного веса).
\end{enumerate}


\subsection{Приведите примеры типовых тестов и контрольных упражнений, которые используются для контроля: ``максимальной силы'', ``взрывной силы'', ``силовой выносливости''.}

В практике физического воспитания количественно силовые возможности оцениваются двумя способами:
\begin{enumerate}
    \item с помощью измерительных устройств — динамометров, динамографов, тензометрических силоизмерительных устройств;
    \item с помощью специальных контрольных упражнений, тестов на силу.
\end{enumerate}

Современные измерительные устройства позволяют измерять силу практически всех мышечных групп в стандартных заданиях (сгибание и разгибание сегментов тела), а также в статических и динамических усилиях (измерение силы действия спортсмена в движении).

В массовой практике для оценки уровня развития силовых качеств наиболее часто используются специальные контрольные упражнения (тесты). Их выполнение не требует какого-либо специального дорогостоящего инвентаря и оборудования. Для определения максимальной силы используют простые по технике выполнения упражнения, например, жим штанги лежа, приседание со штангой и т.п. Результат в этих упражнениях в очень малой степени зависит от уровня технического мастерства. Максимальная сила определяется по наибольшему весу, который может поднять занимающийся (испытуемый).

Для определения уровня развития скоростно-силовых способностей и силовой выносливости используются следующие контрольные упражнения:
\begin{enumerate}
    \item прыжки через скакалку,
    \item подтягивания,
    \item отжимания на параллельных брусьях от пола или от скамейки,
    \item поднимание туловища из положения лежа с согнутыми коленями,
    \item висы на согнутых и полусогнутых руках,
    \item подъем переворотом на высокой перекладине,
    \item прыжок в длину с места с двух ног,
    \item тройной прыжок с ноги на ногу (вариант — только на правой и только на левой ноге),
    \item поднимание и опускание прямых ног до ограничителя,
    \item прыжок вверх со взмахом и без взмаха рук (определяется высота выпрыгивания),
    \item метание набивного мяча (1—3 кг) из различных исходных положений двумя и одной рукой, и др.
\end{enumerate}

Критериями оценки скоростно-силовых способностей и силовой выносливости служат число подтягиваний, отжиманий, время удержания определенного положения туловища, дальность метаний (бросков), прыжков и т.п.


\subsection{Дайте определение скоростных способностей.}

Для характеристики возможностей выполнять двигательные задания с максимальной скоростью в течение ряда лет использовался обобщённый термин «быстрота». Учитывая множественность форм проявления движений и высокую их специфичность, этот термин в последние годы заменили на понятие «скоростные способности».

\textbf{Скоростные способности} --- это комплекс функциональных свойств человека, обеспечивающих выполнение двигательных действий в минимальный для данных условий отрезок времени.


\subsection{Какие разновидности скоростных способностей существуют?}

Различают элементарные и комплексные формы проявления скоростных способностей. К элементарным формам относятся:
\begin{enumerate}
    \item Скорость двигательной реакции;
    \item Скорость одиночного движения;
    \item Частота движений (количество движений в единицу времени).
\end{enumerate}

К комплексным формам проявления скоростных способностей относятся:
\begin{enumerate}
    \item Способность быстро набирать скорость на старте до максимально возможной (стартовый разгон в спринтерском беге, конькобежном спорте, рывки в футболе);
    \item Способность к достижению высокого уровня дистанционной скорости — в беге, плавании и других циклических локомоциях;
    \item Способность быстро переключаться с одних действий на другие и т.п.
\end{enumerate}


\subsection{Перечислите механизмы, обеспечивающие проявление скоростных способностей.}

Проявление форм быстроты и скорости движений зависит от целого ряда факторов:
\begin{enumerate}
    \item состояния центральной нервной системы и нервно-мышечного аппарата человека;
    \item морфологических особенностей мышечной ткани, её композиции (т.е. от соотношения быстрых и медленных волокон);
    \item силы мышц;
    \item способности мышц быстро переходить из напряжённого состояния в расслабленное;
    \item энергетических запасов в мышце (аденозинтрифосфорная кислота — АТФ и креатинфосфат — КТФ);
    \item амплитуды движений, т.е. от степени подвижности в суставах;
    \item способности к координации движений при скоростной работе;
    \item биологического ритма жизнедеятельности организма;
    \item возраста и пола;
    \item скоростных природных способностей человека.
\end{enumerate}

С физиологической точки зрения быстрота реакции зависит от скорости протекания следующих пяти фаз:
\begin{enumerate}
    \item возникновения возбуждения в рецепторе (зрительном, слуховом, тактильном и др.), участвующем в восприятии сигнала;
    \item передачи возбуждения в ЦНС;
    \item перехода сигнальной информации по нервным путям, её анализа и формирования эфферентного сигнала;
    \item проведения эфферентного сигнала от ЦНС к мышце;
    \item возбуждения мышцы и появления в ней механизма активности.
\end{enumerate}

Максимальная частота движений зависит от скорости перехода двигательных нервных центров из состояния возбуждения в состояние торможения и обратно, т.е. она зависит от лабильности нервных процессов.

На быстроту, проявляемую в целостных двигательных действиях, влияют: частота нервно-мышечной импульсации, скорость перехода мышц из фазы напряжения в фазу расслабления, темп чередования этих фаз, степень включения в процесс движения быстро сокращающихся мышечных волокон и их синхронная работа.

С биохимической точки зрения быстрота движений зависит от содержания аденозинтрифосфорной кислоты (АТФ) в мышцах, скорости её расщепления и ресинтеза (восстановления).

Научные исследования свидетельствуют, что двигательные способности существенно зависят от факторов генотипа, например, быстрота простой реакции примерно на 60—88\% определяется наследственностью.

На проявление скоростных способностей также влияет и температура внешней среды. Максимальная скорость движений наблюдается при температуре +20—22°C. При 16°C скорость снижается на 6—9\%.

Наиболее благоприятными периодами для развития скоростных способностей как у мальчиков, так и у девочек считается возраст от 7 до 11 лет. Несколько в меньшем темпе рост различных показателей быстроты продолжается с 11 до 14—15 лет. К этому возрасту фактически наступает стабилизация результатов в показателях быстроты простой реакции и максимальной частоты движений. Целенаправленные воздействия или занятия разными видами спорта оказывают положительное влияние на развитие скоростных способностей: специально тренирующиеся имеют преимущество на 5—20\% и более, а рост результатов может продолжаться до 25 лет.

Половые различия в уровне развития скоростных способностей невелики до 12—13-летнего возраста. Позже мальчики начинают опережать девочек, особенно в показателях быстроты целостных двигательных действий (бег, плавание и т.д.).

Скоростные способности человека очень специфичны. Например, можно обладать хорошим стартовым ускорением и невысокой дистанционной скоростью, и наоборот, а тренировка в быстроте реакции практически не сказывается на частоте движений.


\subsection{Назовите группы упражнений, которые используются для развития скоростных способностей.}

Средствами развития скоростных способностей являются упражнения, выполняемые с предельной либо околопредельной скоростью (т.е. скоростные упражнения). Их можно разделить на три основные группы.

\begin{enumerate}
    \item Упражнения, направленно воздействующие на отдельные компоненты скоростных способностей:
          \begin{itemize}
              \item быстроту реакции;
              \item скорость выполнения отдельных движений;
              \item улучшение частоты движений;
              \item улучшение стартовой скорости;
              \item скоростную выносливость;
              \item быстроту выполнения последовательных двигательных действий в целом (например, бега, плавания, ведения мяча).
          \end{itemize}

    \item Упражнения комплексного (разностороннего) воздействия на все основные компоненты скоростных способностей (например, спортивные и подвижные игры, эстафеты, единоборства и т.д.).

    \item Упражнения сопряженного воздействия:
          \begin{itemize}
              \item на скоростные и все другие способности (скоростные и силовые, скоростные и координационные, скоростные и выносливость);
              \item на скоростные способности и совершенствование двигательных действий (в беге, плавании, спортивных играх и др.).
          \end{itemize}
\end{enumerate}

В спортивной практике для развития быстроты отдельных движений применяются те же упражнения, что и для развития взрывной силы, но без отягощения или с таким отягощением, которое не снижает скорости движений. Кроме этого используются такие упражнения, которые выполняют с неполным размахом, с максимальной скоростью и с резкой остановкой движений, а также старты и спурты.

Для развития частоты движений применяются:
\begin{itemize}
    \item циклические упражнения в условиях, способствующих повышению темпа движений;
    \item бег под уклон за мотоциклом, с тяговым устройством;
    \item быстрые движения ногами и руками, выполняемые в высоком темпе за счет сокращения размаха, а затем постепенного его увеличения;
    \item упражнения на повышение скорости расслабления мышечных групп после их сокращения.
\end{itemize}

Для развития скоростных возможностей в их комплексном выражении применяются три группы упражнений:
\begin{itemize}
    \item упражнения, которые используются для развития быстроты реакции;
    \item упражнения, которые используются для развития скорости отдельных движений, в том числе для передвижения на различных коротких отрезках (от 10 до 100 м);
    \item упражнения, характеризующиеся взрывным характером.
\end{itemize}


\subsection{Приведите примеры типовых тестов и контрольных упражнений, которые используются для контроля: ``простой реакции'', ``сложной реакции'', ``скорости одиночного движения'', ``максимальной частоты движений в разных суставах'', ``скорости, проявляемой в целостных двигательных действиях''.}

Контрольные упражнения (тесты) для оценки скоростных способностей делятся на четыре группы:
\begin{enumerate}
    \item Контрольные упражнения для оценки простой реакции.
          Время простой реакции измеряют в условиях, когда заранее известен и тип сигнала, и способ ответа.
          Время реакции на свет, звук, прикосновение определяется с помощью различных реакциометров,
          измеряющих время реакции с точностью до 0,01 или 0,001 с.
          Для оценки простой реакции используют не менее 10 попыток и определяют среднее время реагирования.
    \item Контрольные упражнения для оценки сложной реакции.
          Сложная реакция характеризуется тем, что тип сигнала и вследствие этого способ ответа неизвестны
          (реакции в играх и единоборствах). В лабораторных условиях время выбора измеряют с помощью слайдов с игровыми или
          боевыми ситуациями, которые предлагаются испытуемому. Оценив ситуацию, последний реагирует либо нажатием кнопки,
          либо словесным ответом, либо специальным действием.
    \item Контрольные упражнения для оценки скорости одиночных движений.
          Время удара, передачи мяча, броска, одного шага и т.п. определяют с помощью биомеханической аппаратуры.
    \item Контрольные упражнения для оценки максимальной частоты движений.
          Частоту движений ног, рук оценивают с помощью теппинг-тестов. Регистрируется число движений за 5-20 с.
    \item Контрольные упражнения для оценки скорости, выполняемой в целостных двигательных действиях.
          Бег на 30, 50, 60, 100 метров на скорость преодоления дистанции.
\end{enumerate}


\subsection{Опишите методики развития основных компонентов скоростных способностей (скорость простой и сложной двигательной реакции, скорости одиночного движения, комплексных форм проявления скоростных способностей).}

\subsection*{Методика развития быстроты двигательных реакций}

\textbf{Простая реакция} - это ответ заранее известным движением на заранее известный, но внезапно появляющийся сигнал
(зрительный, слуховой, тактильный).

Примерами такого вида реакций являются начало двигательного действия (старт) в ответ на выстрел стартового пистолета в
легкой атлетике или в плавании, прекращение игры при свистке арбитра и т.п. Быстрота простой реакции определяется по так
называемому латентному (скрытому) периоду реакции — временному отрезку от момента появления сигнала до момента начала движения.

Основной метод при развитии быстроты реакции — метод повторного выполнения упражнения. Он заключается в повторном реагировании
на внезапно возникающий (заранее обусловленный) раздражитель с установкой на сокращение времени реагирования.

Упражнения на быстроту реакции вначале выполняют в облегченных условиях (учитывая, что время реакции зависит от сложности
последующего действия, ее отрабатывают отдельно, вводя облегченные исходные положения и т.д.). Например, в легкой атлетике
(в беге на короткие дистанции) отдельно упражняются в скорости реакции на стартовый сигнал с опорой руками о какие-либо предметы
в положении высокого старта и отдельно без стартового сигнала в быстроте выполнения первых беговых шагов.

Как правило, реакция осуществляется не изолированно, а в составе конкретно направленного двигательного действия или его элемента
(старт, атакующее или защитное действие, элементы игровых действий и т.п.). Поэтому для совершенствования быстроты
простой двигательной реакции применяют упражнения на быстроту реагирования в условиях, максимально приближенных к
соревновательным, изменяют время между предварительной и исполнительной командами (вариативные ситуации).

Чтобы избежать чрезмерной стабилизации быстроты простой реакции, необходимо использовать, особенно с детьми школьного возраста,
игровой метод, который предполагает выполнение заданий в условиях постоянного и случайного изменения ситуаций.

Простые реакции обладают свойством переноса: если человек быстро реагирует на сигналы в одной ситуации, то он будет быстро
реагировать на них и в других ситуациях.

\textbf{Сложные двигательные реакции} встречаются в видах спорта, характеризующихся постоянной и внезапной сменой ситуации
действий (спортивные игры, единоборства, горнолыжный спорт и т.д.). В сложных реакциях выделяют: реакцию на движущийся объект
(мяч, шайба и т.п.) и реакцию «выбора» (когда из нескольких возможных действий требуется мгновенно выбрать одно,
адекватное данной ситуации).

Период реакции на движущий объект складывается из четырех элементов:
\begin{enumerate}
    \item Человек должен увидеть движущий объект (мяч, игрока).
    \item Оценить направление и скорость его движения.
    \item Выбрать план действий.
    \item Начать его осуществление.
\end{enumerate}

Основная доля этого времени (более 80\%) уходит на зрительное восприятие, т.е. на умение увидеть предмет.
Для тренировки этой способности используются упражнения, при выполнении которых следует:
\begin{itemize}
    \item постоянно увеличивать скорость движения объекта;
    \item сокращать дистанцию между объектом и занимающимся;
    \item уменьшать размеры движущегося объекта.
\end{itemize}

\textbf{Реакции выбора} связана с выбором двигательного ответа из нескольких возможных. Время реакции выбора во многом зависит
от большого запаса тактических действий и технических приемов.

Для развития быстроты реакции с выбором следует:
\begin{enumerate}
    \item Постепенно усложнять характер ответных действий и условия их выполнения. Например, сначала обучают выполнять защиту
          в ответ на заранее обусловленный удар, затем ученику предлагают реагировать на одну из двух возможных атак, затем трех и т.д.
    \item Развивать способность предугадывать действия противника. Другими словами, реагировать не столько на соперника или
          партнера, сколько на малозаметные движения (осанку, мимику, эмоциональное состояние и т.п.)
\end{enumerate}

\subsection*{Методика развития скорости одиночного движения и частоты движения}

Быстрота одиночного движения проявляется в способности с высокой скоростью выполнять отдельные двигательные акты.
Это, например, скорость движения ноги при ударе по футбольному мячу, скорость движения руки при ударе по волейбольному мячу или
при метании копья.

Наибольшая быстрота одиночного движения достигается при отсутствии добавочного внешнего сопротивления. С увеличением
внешнего сопротивления повышение скорости движений достигается за счет повышения мощности проявляемых при этом усилий.
Последняя определяется взрывными способностями мышц. В данном случае развитие быстроты одиночного движения целесообразно
проводить совместно с развитием силовых способностей, используя упражнения с отягощениями (утяжеленные перчатки у боксера,
утяжеленную обувь в прыжках и т.п.). Но подобные упражнения следует применять лишь после того, как будет хорошо освоена
техника основного навыка без отягощения.

Наряду с усложнением условий используют также облегченные условия:
\begin{itemize}
    \item «уменьшают» вес тела занимающегося за счет приложения внешних сил (например, непосредственная помощь
          преподавателя или партнера с применением подвесных лонж и без них (в гимнастических упражнениях));
    \item ограничивают сопротивление естественной среды (например, бег по ветру, плавание по течению и т.п.);
    \item используют внешние условия, помогающие занимающемуся произвести ускорение за счет инерции движения своего тела
          (бег под гору, бег по наклонной дорожке и т.п.).
\end{itemize}

Эффективным методом является контрастный (вариативный) метод, предполагающий чередование выполнения скоростных упражнений
в затрудненных, обычных и облегченных условиях.

В циклических видах физических упражнений (бег, плавание и др.) скорость передвижения зависит от оптимального соотношения
длины шага и темпа движений (частота движений в единицу времени). Оба эти показателя тренируемы, но каждый имеет свою природу и
методику развития. Если длина шага определяется силовыми и скоростно-силовыми показателями, то темп отражает скоростную
характеристику. Темп зависит от состояния нервной системы, возможностей опорно-двигательного аппарата, способности мышц к
расслаблению и переключению.

Для повышения темпа используются следующие методические приемы:
\begin{itemize}
    \item повторное выполнение циклических упражнений с максимальной частотой шагов;
    \item повторное выполнение циклических упражнений с различной частотой шагов и фиксированием времени;
    \item игры и эстафеты;
    \item упражнения на расслабление.
\end{itemize}

\subsection*{Методика развития комплексных форм проявления скоростных способностей}

Большая часть двигательных действий требует проявления всех скоростных способностей (в беге, прыжках, ведении и броске мяча и др.).
Методы тренировки включают в себя не только раздельное развитие скоростных способностей, но и комплексное их применение.

При развитии комплексных скоростных способностей ведущим является повторный метод с проявлением в упражнениях максимальной
скорости и «полным» интервалом отдыха между ними.

Не менее важное значение имеет игровой и соревновательные методы, использование которых создает дополнительный стимул для
предельного проявления скоростных способностей за счет повышения интереса, мотивации, эмоционального подъема.

Эффективным методом повышения скоростных способностей является вариативный метод, предполагающий чередование скоростных
упражнений в затрудненных, облегченных и обычных условиях.

В практике нередко приходится наблюдать остановку в росте результатов — «скоростной барьер». Одной из причин этого явления
следует считать применение одних и тех же методов, методических приемов, средств и условий занятий. В результате возникают
условия к образованию двигательного динамического стереотипа, т.е. стойкой системности нервных процессов в коре
больших полушарий головного мозга. Это приводит к стабилизации скоростных параметров движений.

Для предупреждения «скоростного барьера» на занятиях с детьми не следует спешить с узкой специализацией и использовать
средства и методы при их широкой вариативности.

Для преодоления скоростного барьера создают облегченные условия, в которых бы спортсмен превысил свою наивысшую скорость.
В ряде случаев целесообразным оказывается прекращение на некоторое время занятий в избранном виде спорта и переключение на иные
виды физических упражнений, с помощью которых можно повысить уровень скоростных способностей.


\subsection{Дайте определение понятий ``выносливость'', ``общая выносливость'', ``специальная выносливость''.}

\textbf{Выносливость} – способность человека противостоять физическому утомлению в процессе мышечной деятельности.

В практике физической культуры выделяют общую и специальную выносливость. Под \textbf{общей} понимают выносливость к
продолжительной работе умеренной интенсивности, включающей функционирование всего мышечного аппарата.
Человек, который может выдержать длительный бег в умеренном темпе, может выполнять и другую работу в таком же темпе
(ходьба на лыжах). Основными компонентами общей выносливости являются возможности аэробной системы энергообеспечения,
функциональная и биомеханическая экономизация. Общая выносливость является предпосылкой для развития специальной выносливости.

Выносливость по отношению к определенной деятельности, выбранной как предмет специализации, называют
\textbf{специальной} (например, специальная выносливость бегуна, боксера, игровика).


\subsection{Какие разновидности выносливости существуют?}

Специальная выносливость классифицируется:
\begin{itemize}
    \item По признакам двигательного действия, с помощью которого решается двигательная задача (прыжковая выносливость);
    \item По признакам двигательной деятельности, в условиях которой решается двигательная задача (игровая выносливость);
    \item По признакам взаимодействия с другими физическими качествами, необходимыми для решения двигательной задачи
          (силовая выносливость, скоростная выносливость, координационная выносливость).
\end{itemize}

Специальная выносливость зависит от возможностей нервно-мышечного аппарата, быстроты расходования ресурсов внутримышечных
источников энергии, от техники владения двигательными действиями и от уровня развития других двигательных способностей.

В зависимости от преимущественного проявления других способностей выделяют:
\begin{itemize}
    \item скоростную выносливость;
    \item силовую выносливость;
    \item координационную выносливость.
\end{itemize}

В зависимости от мощности (интенсивности) работы выделяют:
\begin{itemize}
    \item выносливость к работе умеренной мощности;
    \item выносливость к работе большой мощности;
    \item выносливость к работе субмаксимальной мощности;
    \item выносливость к работе максимальной мощности.
\end{itemize}

Выше перечислены основные и наиболее исследованные виды выносливости. Но в практике физической культуры существуют и
другие проявления выносливости, которые группируются по тем или иным признакам, например:
\begin{itemize}
    \item выносливость к работе циклического, ациклического и смешанного характера;
    \item выносливость статическая и динамическая;
    \item выносливость аэробная и анаэробная;
    \item выносливость дистанционная, игровая или многоборная;
    \item выносливость локальная, региональная или глобальная.
\end{itemize}

Различные виды выносливости независимы или мало зависимы друг от друга. Например, можно обладать высокой силовой выносливостью,
но недостаточно скоростной.


\subsection{Перечислите механизмы, обеспечивающие проявление выносливости.}

Уровень развития и проявление выносливости зависит от следующих факторов:
\begin{enumerate}
    \item \textbf{Биоэнергетические факторы} включают объем энергетических ресурсов, которым располагает организм, и
          функциональные возможности его систем (дыхания, сердечно-сосудистой системы, системы крови).
          Образование энергии, необходимой для работы на выносливость, происходит в результате химических превращений.
          Основными источниками энергообеспечения являются аэробные, анаэробные алактатные (продолжительность работы до 20 сек),
          анаэробные гликолитические (продолжительность работы от 20 секунд до 5-6 минут).
    \item \textbf{Факторы функциональной и биохимической экономизации} определяют соотношение результата выполнения
          упражнения и затрат на его достижение. Экономизация имеет две стороны:
          \begin{itemize}
              \item механическую, зависящую от уровня владения техникой упражнения;
              \item физиолого-биохимическую, которая определяется тем, какая доля работы выполняется за счет энергии
                    окислительной работы без накопления молочной кислоты.
          \end{itemize}
          Причем, чем выше квалификация спортсмена, тем выше экономичность выполняемой им работы на выносливость.
    \item \textbf{Факторы функциональной устойчивости} позволяют сохранить активность функциональных систем организма при
          неблагоприятных сдвигах в его внутренней среде, вызванной работой (например, кислородного долга, увеличение
          концентрации молочной кислоты в крови). От функциональной устойчивости зависит способность человека сохранять
          заданные технические и тактические параметры деятельности, несмотря на нарастающее утомление.
    \item \textbf{Личностно-психические факторы} оказывают большое влияние на проявление выносливости, особенно в сложных
          условиях. К ним относят мотивацию на достижение наивысших результатов, а также такие волевые качества как
          настойчивость, выдержка, целеустремленность и умение терпеть неблагоприятные сдвиги во внутренней среде организма.
    \item \textbf{Наследственные факторы}. Генетический фактор в большей степени существенно воздействует на развитие анаэробных
          возможностей, статической выносливости и в меньшей степени на аэробные.
    \item Среди других факторов, оказывающих влияние на выносливость человека, следует выделить возраст, пол, морфологические
          особенности человека и условия деятельности.
\end{enumerate}


\subsection{Какие средства наиболее эффективны для развития общей выносливости, силовой выносливости, скоростной выносливости?}

Средствами развития общей (аэробной) выносливости являются упражнения, вызывающие максимальную производительность
сердечно-сосудистой и дыхательной систем. В практике физической культуры применяют самые разнообразные физические упражнения
циклического и ациклического характера (например, бег, плавание, езда на велосипеде и др.).

Основные требования, предъявляемые к ним следующие:
\begin{itemize}
    \item упражнения должны выполняться в зонах умеренной и большой мощности работы;
    \item их продолжительность от нескольких минут до 60-90 минут;
    \item работа осуществляется при глобальном функционировании мышц.
\end{itemize}

Большинство видов специальной выносливости в значительной мере обусловлено уровнем развития анаэробных возможностей,
для чего используют упражнения, включающие функционирование большой группы мышц и позволяющие выполнять работу с
предельной и околопредельной интенсивностью.

При выполнении большинства физических упражнений на развитие выносливости суммарная нагрузка на организм достаточно
полно характеризуется следующими компонентами:
\begin{itemize}
    \item интенсивностью упражнения,
    \item продолжительностью упражнения,
    \item числом повторений,
    \item продолжительностью интервалов отдыха,
    \item характером отдыха.
\end{itemize}

Определять конкретные параметры нагрузки и отдыха необходимо каждый раз при выборе того или иного метода.


\subsection{Приведите примеры типовых тестов и контрольных упражнений, которые используются для контроля за развитием видов выносливости.}

О степени развития выносливости можно судить на основе двух групп показателей:
\begin{enumerate}
    \item \textbf{Внешних}, которые характеризуют результативность двигательной деятельности человека во время утомления.
    \item \textbf{Внутренних}, которые отражают определенные изменения в функционировании различных органов и систем
          организма, обеспечивающих выполнение данной деятельности (изменения в ЦНС, сердечно-сосудистой, дыхательной,
          эндокринной и других системах человека).
\end{enumerate}

Внешние показатели выносливости в циклических упражнениях могут быть следующие показатели:
\begin{itemize}
    \item пройденная дистанция в заданное время (например, в «часовом беге» или в 12-минутном тесте Купера);
    \item минимальное время преодоления достаточно протяженной дистанции (например, бег на 500 м, плавание на 1500 м);
    \item наибольшая дистанция при передвижении с заданной скоростью «до отказа».
\end{itemize}

В силовых упражнениях выносливость характеризуется:
\begin{itemize}
    \item числом возможных повторений этого упражнения (предельным количеством подтягиваний, приседаний на одной ноге);
    \item предельным временем сохранения позы тела или наименьшим временем выполнения силовых упражнений (например,
          при лазанье по канату или 6-разовом подтягивании);
    \item наибольшим числом движений в заданное время (например, присесть как можно больше в течение 10 секунд).
\end{itemize}

При любых физических упражнениях внешним показателем выносливости человека являются величина и характер изменений различных
биомеханических параметров двигательного действия (длина, частота шагов, время отталкивания, точность движения и др.)
в начале, середине и в конце работы. Сравнивая их значения в разные периоды времени, определяют степень различия и дают
заключение об уровне выносливости.
