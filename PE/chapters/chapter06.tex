\section{Семестр}

\subsection{Определение понятий «гибкость», «активная гибкость», «пассивная гибкость».}



\subsection{Виды гибкости.}



\subsection{Механизмы, обеспечивающие проявление гибкости.}



\subsection{Группы упражнений, которые используются для развития гибкости.}



\subsection{Методы развития гибкости.}



\subsection{Типовые тесты и контрольные упражнения, которые используются для контроля гибкости.}



\subsection{Определение понятий «координация», «ловкость», «координационные способности».}



\subsection{Виды координационных способностей.}



\subsection{Факторы, определяющие развитие координационных способностей.}



\subsection{Возрастно-половые и индивидуальные особенности развития координационных способностей.}



\subsection{Требования к физическим упражнениям, используемым для развития координационных способностей.}



\subsection{Группы упражнений, которые используются для развития координационных способностей.}



\subsection{Методы развития координационных способностей.}



\subsection{Методические приемы при использовании метода строго регламентированного упражнения.}



\subsection{Методика развития координационных способностей, основанных на проприоцептивной чувствительности.}



\subsection{Методика развития способности к ориентированию в пространстве.}



\subsection{Средства и методы для развития чувства ритма.}



\subsection{Методика развития способности к статическому и динамическому равновесию.}



\subsection{Группы физических упражнений для борьбы с координационной напряженностью.}



\subsection{Критерии оценки координационных способностей.}



\subsection{Методы оценки уровня развития координационных способностей.}



\subsection{Типовые тесты и контрольные упражнения для контроля координационных способностей.}
