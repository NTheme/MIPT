\section{Семестр}

\subsection{Что понимается под физической нагрузкой?}



\subsection{Какие стороны имеет нагрузка? Охарактеризуйте каждую.}



\subsection{Характеристика взаимосвязи внешней и внутренней сторон нагрузки.}



\subsection{Факторы, с помощью которых можно регулировать нагрузку.}



\subsection{Разновидности отдыха в зависимости от его продолжительности.}



\subsection{Разновидности отдыха в зависимости от его характера.}



\subsection{Краткая характеристика понятий «метод», «методический прием», «методика», «методический подход» и практическое соотношение между ними.}



\subsection{Группы методов в теории физической культуры.}



\subsection{Методы формирования знаний.}



\subsection{Классификация методов, применяемых при обучении двигательным действиям.}



\subsection{Классификация методов, связанных с нормированием и управлением параметрами нагрузки в процессе выполнения упражнения.}



\subsection{Преимущества и недостатки игрового и соревновательного методов.}



\subsection{Что понимается под принципами в теории физической культуры?}



\subsection{Социальные принципы формирования физической культуры человека.}



\subsection{Общеметодические принципы в теории физической культуры.}



\subsection{Принцип сознательности и активности.}



\subsection{Формы проявления наглядности.}



\subsection{Принцип доступности и индивидуализации.}



\subsection{Принцип непрерывности в физической культуре.}



\subsection{Значение реализации принципа прогрессирования воздействий.}



\subsection{Сущность принципа цикличности.}



\subsection{Значение принципа возрастной адекватности педагогического воздействия в физической культуре.}
